\documentclass[11pt]{article}
\usepackage[sc]{mathpazo} %Like Palatino with extensive math support
\usepackage{fullpage}
\usepackage[authoryear,sectionbib,sort]{natbib}
\linespread{1.7}
\usepackage[utf8]{inputenc}
\usepackage{lineno}
\usepackage{titlesec}
\titleformat{\section}[block]{\Large\bfseries\filcenter}{\thesection}{1em}{}
\titleformat{\subsection}[block]{\Large\itshape\filcenter}{\thesubsection}{1em}{}
\titleformat{\subsubsection}[block]{\large\itshape}{\thesubsubsection}{1em}{}
\titleformat{\paragraph}[runin]{\itshape}{\theparagraph}{1em}{}[. ]\renewcommand{\refname}{Literature Cited}
\newcommand{\tom}[2]{{\color{red}{#1}}\footnote{\textit{\color{red}{#2}}}}
\newcommand{\ali}[2]{{\color{pink}{#1}}\footnote{\textit{\color{pink}{#2}}}}  

%%%%%%%%%%%%%%%%%%%%%
% Line numbering
%%%%%%%%%%%%%%%%%%%%%
%
% Please use line numbering with your initial submission and
% subsequent revisions. After acceptance, please turn line numbering
% off by adding percent signs to the lines %\usepackage{lineno} and
% to %\linenumbers{} and %\modulolinenumbers[3] below.
%
% To avoid line numbering being thrown off around math environments,
% the math environments have to be wrapped using
% \begin{linenomath*} and \end{linenomath*}
%
% (Thanks to Vlastimil Krivan for pointing this out to us!)

\title{Thank you, next: partner turnover elevates benefits of mutualism for an ant-tended plant}

% This version of the LaTeX template was last updated on
% May 11, 2023.

%%%%%%%%%%%%%%%%%%%%%
% Authorship
%%%%%%%%%%%%%%%%%%%%%
% Please remove authorship information while your paper is under review,
% unless you wish to waive your anonymity under double-blind review. You
% will need to add this information back in to your final files after
% acceptance.

\author{Alexandra Campbell$^{1,\ast}$ \\ 
	Tom E.X. Miller$^{1}$}

\date{}

\begin{document}
	
	\maketitle
	
	\noindent{} 1. Program in Ecology and Evolutionary Biology, Department of BioSciences, Rice University, Houston, Texas 77005;
	
	\noindent{} $\ast$ Corresponding author; e-mail: amc49@rice.edu.
	
	
	\textit{Manuscript elements}: 
	
	\bigskip
	
	\textit{Keywords}: 
	
	\bigskip
	
	\textit{Manuscript type}: Article.
	
	\bigskip
	
	\noindent{\footnotesize Prepared using the suggested \LaTeX{} template for \textit{Am.\ Nat.}}
	
\linenumbers{}
\modulolinenumbers[3]

\newpage{}

\section*{Abstract}


\newpage{}

\section*{Introduction}

% The journal does not have numbered sections in the main portion of
% articles. Please refrain from using section references (à la
% section~\ref{section:CountingOwlEggs}), and refer to sections by name
% (e.g. section ``Counting Owl Eggs'').

Mutualisms are species interactions where all participants benefit, leading to higher individual fitness and increased population growth rates. 
They are among the most widespread species interactions \citep{Bronstein1994,Chamberlain2014,Frederickson2013}, but can deteriorate into commensalism or parasitism  \citep{Rodriguez-Rodriguez2017,Song2020,Mandyam2014,Thrall2007, Bahia2022}.
%One reason they are so ubiquitous is they take many forms in nature, including defense for food or housing \cite{Willmer1997}, pollination and dispersal for food \cite{Sakai2002,Burns2004}, resource uptake for housing \cite{Holland2010}, etc. 
Mutualisms are considered more context dependent than other species interactions \citep{Chamberlain2014,Frederickson2013}, meaning the magnitude and sign of interaction strength are often determined by environmental conditions and species' identities.

Mutualism is defined at the level of a species pair (+/+), but these interactions are embedded within multi-species communities. 
Growing evidence suggests that pairwise interactions are poor predictors of the net effects of multi-species mutualism \cite{Afkhami2014,Palmer2010}. 
A focal mutualist may interact with multiple guilds of partner types (e.g., plants that interact with pollinators, seed dispersers, soil microbes, and ant defenders) or with multiple partner species within the same guild (e.g., plants visited by multiple pollinator species). 
Even within a mutualist guild, partner species often differ in the amount or type of goods or services they provide, making partner identity an important source of contingency in mutualism \cite{Stanton2003}. 
Whether and how partner diversity modifies the demographic effects of mutualistic interactions remain open questions within relevance in applied settings such as agriculture \citep{rogers2014bee}, restoration (cite), and pest management (cite). 

There are multiple mechanisms by which partner diversity can influence the net benefits accrued by a focal mutualist, mirroring the mechanisms by which, at a larger scale of organization, biodiversity can influence ecosystem function (BEF chapter). 
When there is a consistent hierarchy of fitness effects -- a consistent ranking of best to worst mutualists -- a more diverse sample of the partner community may be more likely to include the best partner \cite{Frederickson2013}.
When this leads to the fitness of a focal mutualist interacting with multiple partners being equal to the fitness of a focal mutualist interacting with only the highest quality partner, sampling effect can explain the positive effects of partner diversity \cite{Batstone2018}. 

\section*{Methods}

\section*{Results}

\section*{Discussion}


\section*{Conclusion}

\section*{Acknowledgments}

%%%%%%%%%%%%%%%%%%%%%
% Statement of Authorship
%%%%%%%%%%%%%%%%%%%%%
% This section should also be commented out while your MS is undergoing
% double-blind review. The specifics should of course be adapted to
% your paper, but the paragraph below gives some hints of possible
% contributions.

\section*{Data and Code Availability}

\section*{Appendix A: Additional Methods and Parameters}

% In most cases, authors should typeset supplementary material in a separate,
% author-supplied PDF. For author-supplied PDFs, please consult the
% AmNat_supp_template.tex document, available from
% https://www.journals.uchicago.edu/journals/an/instruct 
%
% By contrast, the Appendix instructions below apply to cases in which
% a brief appendix is to appear in print after the main body of the article.
% That notably includes descriptions of methods, tables defining parameters,
% and other material necessary for reproducing the MS's results.
%
% Please reset counters for the appendix (thus normally figure A1, 
% figure A2, table A1, etc.).
%
% Most AmNat articles have no more than one print appendix. If your article
% has more than one, counters for each appendix should match the letter of
% that appendix. For example, tables in Appendix B should be numbered table B1, % table C2, etc. This applies to tables, equations, and figures.
%
% It's better not to use the \appendix command, because we have some
% formatting peculiarities that \appendix conflicts with.

\renewcommand{\theequation}{A\arabic{equation}}
% redefine the command that creates the equation number.
\renewcommand{\thetable}{A\arabic{table}}
\setcounter{equation}{0}  % reset counter 
\setcounter{figure}{0}
\setcounter{table}{0}

%%%%%%%%%%%%%%%%%%%%%
% Bibliography
%%%%%%%%%%%%%%%%%%%%%
% You can either type your references following the examples below, or
% compile your BiBTeX database and paste the contents of your .bbl file
% here. The amnatnat.bst style file should work for this---but please
% let us know if you run into any hitches with it!
%
% If you upload a .bib file with your submission, please upload the .bbl
% file as well; this will be required for typesetting.
%
% The list below includes sample journal articles, book chapters, and
% Dryad references.
\bibliographystyle{apalike}
\bibliography{References.bib}


\newpage{}

\section*{Tables}
\renewcommand{\thetable}{\arabic{table}}
\setcounter{table}{0}


\newpage{}

\section*{Figure legends}


%%%%%%%%%%%%%%%%%%%%%
% Videos
%%%%%%%%%%%%%%%%%%%%%
% If you have videos, journal style for them is generally similar to that for
% figures. 

%%%%% Include the text below if you have videos



%%%%% Include the above if you have videos


\end{document}
