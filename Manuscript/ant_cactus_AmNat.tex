\documentclass[11pt]{article}
\usepackage{amsmath}
%\usepackage[sc]{mathpazo} %Like Palatino with extensive math support
\usepackage{fullpage}
\usepackage[authoryear]{natbib}
\linespread{1.7}
\usepackage[utf8]{inputenc}
\usepackage{lineno}
\usepackage{titlesec}
\usepackage{xcolor}
\usepackage{graphicx}
\usepackage{placeins}
\usepackage{float}
\usepackage{subfigure}
\newcommand{\tom}[2]{{\color{red}{#1}}\footnote{\textit{\color{red}{#2}}}} 
\newcommand{\ali}[2]{{\color{blue}{#1}}\footnote{\textit{\color{blue}{#2}}}}  
\usepackage{hyperref}


\titleformat{\section}[block]{\Large\bfseries\filcenter}{\thesection}{1em}{}
\titleformat{\subsection}[block]{\Large\itshape\filcenter}{\thesubsection}{1em}{}
\titleformat{\subsubsection}[block]{\large\itshape}{\thesubsubsection}{1em}{}
\titleformat{\paragraph}[runin]{\itshape}{\theparagraph}{1em}{}[.]\renewcommand{\refname}{Literature Cited}


%%%%%%%%%%%%%%%%%%%%%
% Line numbering
%%%%%%%%%%%%%%%%%%%%%
%
% Please use line numbering with your initial submission and
% subsequent revisions. After acceptance, please turn line numbering
% off by adding percent signs to the lines %\usepackage{lineno} and
% to %\linenumbers{} and %\modulolinenumbers[3] below.
%
% To avoid line numbering being thrown off around math environments,
% the math environments have to be wrapped using
% \begin{linenomath*} and \end{linenomath*}
%
% (Thanks to Vlastimil Krivan for pointing this out to us!)

\title{Thank you, next: demographic consequences of partner diversity and turnover in a multi-species ant-plant mutualism}
% title suggestion, trying to move away from focus on IPM

\author{Alexandra Campbell$^{1,\dagger}$ \\ 
	Tom E.X. Miller$^{1,\ast}$}

\date{}

\begin{document}
	
	\maketitle
	
	\noindent{} 1. Program in Ecology and Evolutionary Biology, Department of BioSciences, Rice University, Houston, Texas 77005;
	
	\noindent{} $\dagger$ e-mail: amc49@rice.edu\\
	\noindent{} $\ast$ e-mail: tom.miller@rice.edu
	
	\bigskip
	
	\textit{Keywords}: Integral Projection Model, \textit{Cylindropuntia imbricata}, population fitness, multi-species mutualism, complementarity, sampling effect, portfolio effect
	
	\bigskip
	
	\textit{Manuscript type}: Article.
	
	\bigskip
	
	\noindent{\footnotesize Prepared using the suggested \LaTeX{} template for \textit{Am.\ Nat.}}
	
\linenumbers{}
\modulolinenumbers[3]

\newpage{}

	\section*{Abstract}
% 200 words max
 The diversity of partners in a multi-species mutualism causes varied demographic effects on the population of the focal mutualist which can be explained by several mechanisms: portfolio effect, complementarity, and sampling effect.
\textit{Cylindropuntia imbracata} (tree cholla) produce extrafloral nectar and various ant partners provide defense from herbivores and seed predators. 
We used plant demographic censuses to parameterize a series of Bayesian hierarchical generalized linear vital rate models to determine the impacts of different partners on the focal mutualists. 
We constructed an Integral Projection Model in which we simulate different combinations of ant partners that don’t occur in nature.
The hierarchical models revealed that different ant partners had different impacts on the cholla vital rates. 
Specifically, \textit{C. opuntiae} tended plants had advantages in both growth and survival when small, and large \textit{L. apiculatum} tended plants had floral viability advantages. 
The IPM results revealed that despite these differences in vital rates, the presence of any ant partner is beneficial, the identity and number of partners do not matter.
This suggests that there are no benefits of diversity in this system.
This study highlights that partner diversity does not always increase the benefits a focal mutualist receives.


\newpage{}
\section*{Introduction}
Mutualisms are species interactions where all participants receive net benefits, leading to higher individual fitness and increased population growth rates. 
They are widespread species interactions but can deteriorate into commensalism or parasitism under conditions that elevate costs or dampen benefits \citep{Rodriguez-Rodriguez2017,Song2020,Mandyam2014,Thrall2007, Bahia2022,Bronstein1994,Chamberlain2014,Frederickson2013,Axelrod1981,Leigh2010}.
Mutualisms are considered more context dependent than other species interactions \citep{Chamberlain2014,Frederickson2013}, meaning the magnitude and sign of interaction strength are often determined by environmental conditions and species' identities \citep{Noe1994,Leigh2010}.

Mutualism is defined at the level of a species pair (+/+) but these interactions are embedded within multi-species communities, and growing evidence suggests that pairwise interactions are poor predictors of the net effects of multi-species mutualism \citep{Afkhami2014,Palmer2010,Bascompte2009,Dattilo2014}. 
A focal mutualist may interact with multiple guilds of partner types (e.g., plants that interact with pollinators, seed dispersers, soil microbes, and ant defenders) or with multiple partner species within the same guild (e.g., plants visited by multiple pollinator species). 
Within a mutualist guild, partner species often differ in the amount or type of goods or services they provide, making partner identity an important source of contingency in mutualism \citep{Stanton2003}. 
Whether and how partner diversity modifies the demographic effects of mutualistic interactions remain open questions within relevance in applied settings \citep{rogers2014, Thibaut2012, Frederickson2005, Palmer2010}.

There are multiple mechanisms by which partner diversity can influence the net benefits accrued by a focal mutualist, mirroring the mechanisms by which, at a larger scale of organization, biodiversity can influence ecosystem function \cite{Yeung2006,Barrett2015,Ushio2020}. 
First, when there is a hierarchy of fitness effects (a consistent ranking of best to worst mutualists), a more diverse sample of the partner community may be more likely to include the best partner \cite{Frederickson2013}.
This can lead to an apparent benefit of diversity driven by a sampling effect \cite{Batstone2018} -- but there is no benefit of diversity \emph{per se}, only better and worse partners. 
If partner associations are mutually exclusive (a focal mutualist may engage with only one partner at a time), then partner diversity may impose opportunity costs, leading to negative effects of a diverse mutualist assemblage relative to exclusive association with the single best partner \citep{Miller2007}. 
Second, even within a single mutualist guild, the benefits conferred by alternative partner species can vary in type and not just degree \cite{Stachowicz2005,Bronstein2006,Stanton2003}. 
This can lead to a positive effect of partner diversity through complementarity of alternative functions \cite{Batstone2018}. 
Interference or synergies between partners can make their combined effect different than the expected from the sum of complementary functions \cite{Afkhami2014}. 
Third, partner species can have species-specific responses to environmental variation, either spatially \citep{Ollerton2006} or temporally \citep{Alarcon2008}. 
Multiple partners can therefore act as a 'portfolio' that stabilizes fitness benefits across spatial or temporal heterogeneity, leading to positive effects of partner diversity through the portfolio effect \cite{Batstone2018,Lazaro2022,Horvitz1990}. 

Partner diversity can have different effects depending on whether partners are present simultaneously or sequentially (partner turnover) \citep{Djieto-Lordon2005, Ness2006, Bruna2014,Barrett2015,Ushio2020,Dattilo2014}. 
Sequential associations are likely when alternative partners engage in interference competition for access to a shared mutualist \cite{Kiers2003,Batstone2018,Tgaard2015,Wulff2008}. 
Turnover can happen at different timescales, from minutes to years \citep{Oliveira1999,Horvitz1986}. 
The frequency of partner turnover can impact the level of benefits received by the focal mutualist, particularly if the benefits continue to accumulate with successive turnover (e.g., when sequential partners provide complementary functions) or if they saturate over time \citep{Sachs2004,Fiala1994}.
Directionality of turnover can also influence effects of partner diversity if partner identity changes consistently across ontogeny of a focal mutualist \citep{Fonseca2003,Noe1994,Dejean2008}.
For example, plant susceptibility to enemies can change across life stages \citep{Boege2005,Barton2010}, so the benefits of a diverse guild of defensive mutualists are greatest when more defensive partner species align with more vulnerable life stages \citep{Djieto-Lordon2005,Dejean2008}.

Defensive ant-plant mutualisms -- where plants provide food and/or housing to ants that in turn defend them from enemies -- are widespread interactions that offer valuable model systems for the ecology and evolution of mutualism \citep{Bronstein1998, Bronstein2006}. 
Extrafloral nectar (EFN) -bearing plants can serve as dietary resources that promote ant abundance and colony size \citep{Byk2011, Ness2009, Ness2006, Donald2022}.
In turn, presence of defensive ant partners is often linked to reductions in herbivory  \citep{Trager2010, Rudgers2004} and demographic advantages for the plant partner \citep{Baez2016}.
Defensive ant-plant mutualisms are commonly multi-species, where a guild of ant partner species share, and often compete for, a plant mutualist \citep{Bronstein1998, Beattie1985, Trager2010, Agrawal1998}.
Ant partners can vary in their ability to deter herbivores \citep{Bruna2014}, and visitation by low quality ant partners can prevent visitation by higher quality partners, consequently causing a reduction in fitness through missed opportunity costs \citep{Fraser2001, Frederickson2005}.
Susceptibility to herbivory can also vary significantly throughout the life stages of the plant \citep{Boege2005}, suggesting that the order and timing of successive partners is important to the fitness impacts of the combined partner guild \citep{Barton2010, Boege2005, Fonseca2003}.
Finally, herbivore identity and pressure can vary inter-annually \cite{Wetzel2023}, much like mutualist identity and presence, meaning the threat plants face can vary just as much as the protection they receive due to temporal stochasticity. 
Previous studies have investigated how ant partner diversity affects plant fitness \citep{Palmer2010,Afkhami2014,Fiala1994,Gaume1998,Dattilo2014,Ludka2015}
However, little is known about the combined effects of partner identity, directional partner turnover, and temporal stochasticity, particularly because the necessary long-term data are rarely available. 
	
This study examined the consequences of partner diversity in a food-for-protection mutualism between the tree cholla cactus (\textit{Cylindriopuntia imbricata}), a long-lived EFN-bearing plant, and multiple species of ant partners.
Previous studies have shown that herbivory by specialized insect herbivores negatively affects cactus fitness \cite{Miller2009}, and that ant defense reduces herbivore damage \cite{Miller2007}. 
Tree cholla are tended by two common ant species (\textit{Liometopum apiculatum} and \textit{Crematogaster opuntiae}) and several additional rarer species, all of which collect EFN during foraging visits but their colonies are ground-nesting and not housed by the plants. 
These ant species locally co-occur but individual plants are typically tended by only one species that patrols the plant around-the-clock and maintains control of the plant's nectar resources, usually for an entire growing season \citep{Ohm2014, Donald2022}. 
Switches between partner species, or between vacancy and ant occupancy, commonly occur from one growing season to the next \citep{Miller2007}. 
Prior experiments suggested a hierarchy of mutualist quality, with \textit{L. apiculatum} providing strong anti-herbivore defense and \textit{C. opuntiae} having net negative effects because herbivore deterrence is outweighed by deterrence of pollinators \citep{Miller2007,Ohm2014}. 
However, all of our previous studies in this system have focused on single life stages (adult plants) or vital rates (seed production) and did not integrate the demographic effects of ant defense across the life cycle, which may be essential for understanding net fitness effects of a diverse partner guild \citep[e.g.,][]{Palmer2010}. 
To our knowledge no previous study has incorporated inter-annual stochasticity into models of ant-plant dynamics, which limits our understanding of diversity benefits that may arise through the portfolio effect. 

Here we used a unique long-term data set that allows us to explore mutualistic associations with multiple partner species, longitudinal turnover in partner identity, and how the demographic effects of alternative partners varies across plant size structure and nearly 20 years of inter-annual fluctuations. 
We used this observational data set of plant demography and ant-plant associations, contextualized by previous ant exclusion experiments, to investigate whether and through which mechanism(s) partner diversity affects the fitness benefits of ant visitation. 
Specifically, we asked:
	\begin{enumerate}	
		\item{What are the demographic effects of association with alternative partners and how do these effects fluctuate across years?}
		\item{What are the frequency and direction of partner turnover across the plant life cycle?}	
		\item{What is the net effect of partner diversity on plant fitness, and what mechanism(s) explain(s) this effect?}
	\end{enumerate}
To answer these questions we used a hierarchical Bayesian statistical approach to estimate demographic vital rates for hosts in different states of ant occupancy and to quantify state-dependent partner turnover from the long-term data. 
We then used a stochastic, multi-state integral projection model (IPM) that combines diverse effects on vital rates and pathways of partner turnover to quantify effects of partner diversity on plant fitness. 


\section*{Methods}
\subsection*{Study System}
  
This study was conducted in the Los Pi$\tilde{n}$os mountains, a small mountain chain located on the Sevilleta National Wildlife Refuge, a Long-term Ecological Research site (SEV-LTER) in central New Mexico, USA.
This is an area characterized by steep, rocky slopes, and perennial vegetation including grasses (\textit{Bouteloua eriopoda} and \textit{B. gracilis}), yuccas, cacti, and junipers. 
Tree cholla cacti are common in high Chihuahuan desert habitats of the southwestern USA \citep{Benson1982}. 
These arborescent plants produce cylindrical segments with large spines. 
In the growing season (May to August in New Mexico), the plants initiate new vegetative segments and flowerbuds at the ends of existing segments. 
While most plants produce new segments every season, only those which are reproductively mature produce flowerbuds. 
%Tree cholla generally reach at least 9 years of age before beginning to reproduce \citep{Ohm2014}.
Like other EFN-bearing cacti, tree cholla secrete nectar from specialized glands on young vegetative segments and flowerbuds \citep{Ness2006,Oliveira1999}. 
Flowerbuds produce more and higher-quality EFN than vegetative segments, making reproductive cholla valuable mutualist partners \citep{Miller2014}. 
Smaller, non-flowering cholla produce little to no EFN and are commonly vacant (no ant visitation at the time of our census) \citep{Miller2014}. 

Tree cholla EFN is harvested by various ground-nesting ant species. 
At SEV-LTER, cholla are visited primarily by two species, \textit{Crematogaster opuntiae} and \textit{Liometopum apiculatum}, as well as other rarer species, including \textit{Forelius pruinosus} and unidentified species in the genera \textit{Aphaenogaster} and \textit{Camponotus}.
\textit{L. apiculatum} are the most frequent visitors with $25\% - 60\%$ of tree cholla tended by these ants in a given year, followed by \textit{C. opuntiae} ($5\% - 20\%$) \citep{Donald2022}. 
Between $ 30\% - 80\%$ of cacti are vacant in any given year. 
Workers of different species rarely co-occur on individual plants, likely due to interspecific competition. 
For example, staged introductions of \textit{C. opuntiae} to \textit{L. apiculatum}-tended plants, and vice versa, provoke aggressive responses by residents (A. Cambpell, \textit{personal observation}).
%Each cholla is visited by a single ant species for the duration of a season, and the species of the visitors can change from one season to the next. 
%In Fall, tree cholla stop producing EFN and the ants vacate until the next growing season. 

Several insect herbivores and seed predators specialize on tree cholla \citep{Mann1969}, and defense against these enemies is the main pathway by which ant visitation affects plant demography. 
The Cerambycid beetle \textit{Moneilema appressum} and an unidentified weevil (Coleoptera: Curculionidae) of the genus \textit{Gerstaekeria} feed on vegetative and reproductive structures as adults and their larvae feed internally. 
Two species of cactus bugs, \textit{Narnia pallidicornis} and \textit{Chelinidea vittiger} (Hemiptera: Coreidae), feed on all cholla parts with a preference for flower buds; their damage can induce floral abortion \citep{Miller2006}. 
A seed predator, \textit{Cahela ponderosella} (Lepidoptera: Pyralidae), oviposits in open flowers and larvae eat seeds in developing fruits. 
These consumers can have significant negative impacts on plant fitness and depress population growth \citep{Miller2009}.
Prior experiments showed that ant-tended tree cholla experience less herbivory and seed predation than plants from which ants were excluded \citep{Miller2007,Ohm2014}. 

\subsection*{Data Collection}
This study is based on long-term demographic data spanning 2004 to 2023 at SEV-LTER. 
From 2004 to 2008, we censused 134 plants distributed across three spatial blocks. 
This initial census group was discontinued in 2009, when we established six 30 $\times$ 30-meter plots and tagged all tree cholla within those plots. 
Two additional 30 $\times$ 30-meter plots were added in 2011, and this group of eight plots has since been censused annually through 2023 (with the exception of 2020 due to the pandemic shutdown). 
For all plants, in May or early June of each year we recorded plant survival since the last survey and, for survivors, we recorded height (cm), maximum crown width (cm), and crown width perpendicular to the maximum (cm).
Size measurements were used to calculate plant volume ($cm^3$) based on the volume of an elliptical cone. 
We measured reproduction by counting flowerbuds, and we distinguished between flowerbuds that were viable and aborted. 
We recorded the ant species present (or vacancy if no ants present).
Occurrences of more than one ant species on one plant were rare (less than 5\% of observations), and for the purpose of this analysis we classified the plant as being occupied by the more abundant species. 
Plots were searched for new recruits each year, and these were added to the census.
These data allowed us to link each plant's demographic fate (survival, growth, and reproduction) to its state of ant visitation. 
In total, the data set includes a total of 9,787 observations of 1141 unique individuals across 15 complete transition years (spanning May/June of year $t$ to May/June of year $t+1$) \citep{DataCholla}. 

We used additional, smaller data sets from previously published studies to estimate seed and seed bank parameters. 
Ohm et al. \citeyear{Ohm2014} provide data on the number of seeds per fruit for plants tended by \textit{L. apiculatum}, \textit{C. opuntiae}, or no ants (experimental exclusion), accounting for their effects on pollinator visitation. 
Elderd and Miller \citeyear{elderd2016quantifying} provide data on seed entry to the seed bank and seedling germination and survival rates. 

\subsection*{Multi-state Integral Projection Model}
The demographic data were used to parameterize a multi-state Integral Projection Model (IPM).
IPMs describe population dynamics in discrete time with functions that relate vital rates to continuous state variables, typically size \citep{ellnerbook}. 
While IPMs are a natural choice for populations with continuous size structure, they can also be modified to accommodate a combination of continuous and discrete state variables, as we do here. 
We constructed a stochastic, multi-state IPM that stitches together population structure associated with plant size and ant state, allowing us to determine the individual fitness effects of each ant species and the composite effects of multiple partners, with ant transition dynamics and inter-annual variability modeled explictly. 

Given the low frequency of ant occupancy states other than \textit{L. apiculatum} and \textit{C. opuntiae} (\textless8\% of observations) we combined all other ants into an ``other'' category, such that our multi-state IPM included four ant states: vacant, \textit{L. apiculatum}, \textit{C. opuntiae}, and Other. 
The ``Other'' category was made up of \textit{Forelius pruinosus} (3.5\% of observations), unidentified species belonging to the genera \textit{Camponotus} (0.9\%), \textit{Aphaenogaster} (0.4\%), \textit{Myrmecocystus} (0.08\%), \textit{Tetramorium} (0.02\%), \textit{Brachymyrmex} (0.02\%), and additional ants not identified to genus or species (2.8\%). 

Ant state is included as a predictor variable in IPM sub-models where there are biologically realistic pathways through which ants could impact the outcome of that process. 
For example, ant partners defend cacti from herbivores, and prior experimental work indicates that ant tending can reduce vegetative tissue loss and floral abortion.
Therefore, ant state was included in sub-models for survival, growth, flowerbud viability, and seed number per flowerbud. 
In contrast, we have no reason to expect that ant tending can directly influence the probability of flowering or flowerbud production independently of its influence on plant size, so these sub-models do not include ant state as a predictor variable. 

We modeled the tree cholla life cycle using continuously size-structured plants where number of plants of size $x$  and ant state $a$ in year $t$ ($n(x,a)_{t}$) predicts the number of plants of size $x'$ and ant state $a'$ in year $t+1$ ($n(x',a')_{t+1}$) based on a size- and ant-specific vital rates. 
The models also includes two discrete seed banks ($B^1_{t}$ and $B^2_{t}$) corresponding to 1 and 2-year old seeds. 
Seed bank dynamics are given by:

\begin{linenomath*}
	$$
	B^1_{t+1} = \delta \sum_{a=1}^{A} \int_L^U  \kappa(a) P(x;\pmb{\tau}^P) F(x;\pmb{\tau}^F) V(a;\pmb{\tau}^V_{a}) n(x,a)_{t} dx \\
	$$
	$$
	B^2_{t+1} =  (1 - \gamma_1)B^1_{t}\\
	$$
	\label{eqn:IPM1}
\end{linenomath*}

\noindent %In these equations $x$ and $x'$ indicate the size of a plant in years $t$ and $t+1$ respectively, $a$ and $a'$ indicate the ant partner of a plant in years $t$ and $t+1$.
Functions $P(x;\pmb{\tau}^P)$ and $F(x;\pmb{\tau}^F)$ give the probability of flowering in year $t$ and the number of flowerbuds produced in year $t$, respectively, by plants of size $x$ in year $t$. 
The proportion of flowerbuds that remain viable through fruit set ($V(a;\pmb{\tau}^V_{a})$) and the number of seeds per fruit ($\kappa(a')$) is dependent on ant state $a$. 
The vectors $\pmb{\tau}$ give year-specific deviates (mean zero) and appear in functions for which we can estimate temporal stochasticity from the long-term data; superscripts indicate the corresponding vital rate and, when present, the $a$ subscript indicates that deviates are specific to plants in ant state $a$.
For example, temporal deviates $\pmb{\tau}^V_{a}$ describe better- and worse-than-average years for flowerbud viability and plants in different ant states can fluctuate independently (good years for \textit{L. apiculatum} -occupied plants may not be good years for \textit{C. opuntiae}-occupied plants, for example). 
Seed production is integrated over the size distribution, from the lower $L$ to upper $U$ size limits, and summed over all possible initial ant states ($A=4$) giving total seed production. 
Seeds are multiplied by the probability of escaping post-dispersal seed predation ($\delta$) to give the number of seeds that enter the one-year old seed bank. 
Plants can recruit out of the year-one seed bank with probability $\gamma_1$ or transition to the two-year seed bank with a probability of $1 - \gamma_1$. 
Seeds in the two-year seed bank are assumed to either germinate with probability $\gamma_2$ or die. 

For the above-ground part of the life cycle, the number of plants of size $x'$ and ant state $a'$ in year $t+1$ is given by survival/growth transitions from size $x$ and ant state $a$ in year $t$, plus germination out of the seed banks:
\begin{linenomath*}
	$$
	n(x',a')_{t+1} = (\gamma_1 B^1_{t} + \gamma_2 B^2_{t} ) \eta(x') \omega \rho_{0}(a')  + \\
	$$
	$$
	\sum_{a=1}^{A} \int_L^U S(x,a;\pmb{\tau^S_{a}}) G(x',x,a;\pmb{\tau^G_{a}}) \rho(x,a,a';\pmb{\tau^{\epsilon}}) n(x,a)_t dx \\
	$$
	\label{eqn:IPM2}
\end{linenomath*}

\noindent The first term in Eq. \tom{\ref{eqn:IPM2}}{We should label s. I am not sure why the equation label is not working here and I did not try to figure it out. It is probably something with the linenomath formatting.} estimates the number of individuals recruiting from a one or two-year seed bank to a plant of size $x'$ and ant state $a'$ based on the recruit size distribution $\eta(x')$ and the probability of over-winter seedling survival ($\omega$) from germination (late summer) to the census (May).
This term is multiplied by $\rho_{0}(a')$, which gives the probability that a new recruit has ant state $a'$ ($\sum\rho_{0}(a')=1$). 
The second term represents all possible transitions from size $x$ and ant $a$ to size $x'$ and ant $a'$, conditioned on survival. 
Survival ($S(x,a;\pmb{\tau}^S_{a})$) and growth from size $x$ to $x'$ ($G(x',x,a;\pmb{\tau}^G_{a})$) are both dependent on initial size and ant state. 
As above, these functions include inter-annual variability through year-specific deviates that can vary by ant state ($\pmb{\tau}_{a}$). 
Finally, ant transition function $\rho(a',a,x;\pmb{\tau}^{\rho})$ gives the probability that an individual transitions from ant state $a$ to $a'$ in the next census, conditional on initial size $x$. 
This function includes inter-annual variability through year-specific intercepts which are consistent across initial ant states ($\pmb{\tau}^\rho$).

\subsection*{Statistical modeling and parameter estimation}
We parameterized the IPM using a series of generalized linear mixed models in a hierarchical Bayesian framework. 
Vital rate models included spatial and temporal random effects associated with plot and year variation, respectively (only year variation is used in the IPM), and included plant size (the natural logarithm of volume, $log(cm^3)$; $x,x'$), ant partner state ($a,a'$), or both as fixed-effect predictor variables. 
As in the IPM, our statistical modeling assumed that demographic effects of ant occupancy are limited to survival, growth, and flowerbud viability. 

\paragraph{Growth}
We fit the growth sub-model ($G(x',x,a;\pmb{\tau^G_{a}})$) to data on size in year $t+1$ ($y^G$) using the skewed normal distribution to account for left-skewed size transitions (at some initial sizes, transitions below the expected future size were more common than transitions above it). 
The skew-normal has three parameters corresponding to location ($\hat{G}$), shape ($\sigma$), and scale ($\alpha$):
\begin{linenomath*}
	$$y_i^G \sim Skewed Normal(\hat{G_i},\sigma_i,\alpha_i) $$
	$$\hat{G_i} = \beta^0_{a[i]} + \beta^1_{a[i]} x_i + \beta^2_{a[i]} x_i^2 + u_{year[i],a[i]} + w_{plot[i]} $$
	$$log(\sigma_i)  = \beta^3 + \beta^4 x_i $$
	$$\alpha_i = \beta^5 + \beta^6 x_i$$
	\label{eqn:growth}
\end{linenomath*}
Here, the location parameter for the $i$th observation $\hat{G_i}$ is defined as a second-order polynomial with ant-size interactions because  preliminary analysis found this was an improvement over a linear relationship (note that the location parameter of the skew-normal is not the mean, but the mean can be derived as $\hat{G} + \frac{\sigma\alpha}{\sqrt{1+\alpha^2}} \sqrt{\frac{2}{\pi}}$). 
The year- and ant-specific random effect $u$ (which parameterizes the $\pmb{\tau}^G_{a}$ vectors) and plot-specific random effect $w$ are normally distributed with variances $\sigma^2_{year}$ and $\sigma^2_{plot}$, respectively. 
Parameters $\sigma_i$ and $\alpha_i$  control residual variance and skewness, respectively, and were defined as linear functions of initial size $x_i$ ($\sigma_i$ is strictly positive and was modeled with a log link function). 
We assume growth variance and skewness were not dependent on ant occupancy state. 

\paragraph{Survival}
The survival sub-model ($S(a,x;\pmb{\tau}_{a}^{S})$) estimates the probability of survival from year $t$ to year $t+1$, with fixed effects of size $x$ and ant partner $a$ in year $t$.
We fit this model to the survival data (alive or dead) using a Bernoulli distribution with a similar linear predictor for the probability of survival as in the growth model but with a logit link function and without the second-order influence of size.

\paragraph{Reproduction}
The flowering sub-model ($P(x;\pmb{\tau}^{P})$) estimates the probability of reproducing in year $t$, with fixed effects size $x$ in year $t$ and random effects of plot and year.
We fit this model to the reproductive status data (vegetative or flowering) using a Bernoulli distribution and a logit link function, similar to the survival model above but with no ant effects.  
The flowerbud function $F(x;\pmb{\tau}^{F})$ estimates the total flowers produced by a reproducing plant in year $t$, with fixed effects of size $x$ in year $t$. 
We fit this model to flowerbud count data (sum of viable and aborted buds) using a zero-truncated negative binomial distribution with a log link and normally distributed year and plot random effects.

The flowerbud viability sub-model ($V(a;\pmb{\tau}^{V}_{a})$) estimates the probability that flowerbuds produced in year $t$ remain viable (not aborted), with fixed effects of ant partner $a$ in year $t$.
We fit this model to floral viability data using a binomial distribution where trials and successes are given by the total number of flowerbuds and the number that are viable, respectively.
This model used a logit link function and included random effects for plot and year following the same structure as the growth and survival models, with ant-specific year random effects. 

Estimates for the number of seeds per fruit were obtained from a field experiment which excluded ants (cite Tom's data).
This data only included  \textit{L. apiculatum} and \textit{C. opuntiae} ants, so we had to make an assumption about the seeds per fruit on plants tended by Other ants.
We chose to use the vacancy estimates for plants tended by other ants, a decision which does not have a significant impact on the final results. 
Additional reproductive parameters for the number of seeds per fruit, probability of entry to the seed bank, germination rates, and recruit size were estimated following methods described in Appendix XX.


\paragraph{Ant Transitions}
The ant transition model ($\rho(x,a,a';\pmb{\tau}^{\epsilon})$) estimates the probability of a cactus being occupied by ant partner $a'$ in year $t+1$ given that it was occupied by initial ant partner $a$  in year $t$, with fixed effect of initial size $x$.
We fit this model to ant partner data using a multinomial distribution with a logit link function. 

\paragraph{Parameter estimation}
We fit models using Stan run through version 4.0.2 of R \cite{Rcite,Rstancite}. 
We used vague priors for all parameters. 
For each model, we obtained three chains of 10,000 iterations, discarding the first 1,500 iterations. 
We visually assessed parameter convergence between and within chains (Appendix \ref{appendix:D}: Figures \ref{fig:Surv_post} -- \ref{fig:Pre_Surv_post} b) and assessed overall model fit with posterior predictive checks to examine how well the fitted model can generate simulated data similar to the real data (Appendix \ref{appendix:D}: Figures \ref{fig:Surv_post} -- \ref{fig:Pre_Surv_post} a).

\subsection*{IPM Analysis}
Analyzing the IPM required that we discretize the continuous IPM kernel into an approximating matrix. 
Size variable $x$ was discretized into $b$ bins, resulting in a $b \times b$ matrix.
In our model there is additional complexity in the form of transitions between $A$ ant states and two additional discrete states (year one and year two seed banks), leading to a matrix size of $A(b+2) \times A(b+2)$.
We used $b = 500$ bins, which we found to be sufficient for numerically stable outputs, and extended the integration limits beyond the minimum ($L$) and maximum ($U$) observed sizes to avoid unintentional eviction using the ``floor-and-ceiling'' method \cite{Williams2012}. 

For stochastic analyses, we estimated the approximating matrix corresponding to each $t$ to $t+1$ transition year. 
To estimate population mean fitness in a stochastic environment where ants responded to annual variation uniquely \tom{($\lambda_{NS}$)}{Let's discuss this symbology.} we simulated population dynamics for 500 years by randomly sampling among the 16 annual transition matrices, discarding the first 100 years of the simulation to minimize the influence of initial conditions. 
Sampling observed transition matrices (rather than independently sampling regression coefficients) produces demographic time series that realistically capture inter-annual variation by preserving correlations between vital rates \cite{metcalf2015statistical}.
We tallied the total population size at each time step as  $N_{t} = B^1_{t} + B^2_{t} + \sum_{a=1}^{A}\int n(x,a)_{t}dx$ and calculated the stochastic growth rate as $log(\lambda_S) = E[log(\frac{N_{t+1}}{N_{t}})]$ \citep{Mark2009}.
We propagated uncertainty from the vital rate models using 100 draws from the joint posterior distribution of model parameters, resulting in a posterior distribution of $\lambda_{NS}$ and other derived quantities.

\subsubsection*{Partner diversity simulation experiments}
Using the fully parameterized multi-state IPM, we conducted simulation experiments to quantify how diversity and identity of ant partners influenced plant fitness in a stochastic environment ($\lambda_{NS}$). 
From the full version of the model (described above) corresponding to the observed assemblage of partners and observed frequencies of partner transition, we created treatments corresponding to all eight ``counter-factual'' scenarios of diversity and composition: no ant partners (complete vacancy); one ant partner (\textit{C. opuntiae} only, \textit{L. apiculatum} only, Other only); two partners (all pairwise combinations of \textit{C. opuntiae}, \textit{L. apiculatum}, and Other); and three partners (observed scenario of all ant states).
These simulation experiments were made possible by extrapolating ant-specific demographic performance across the size distribution, even for combinations of size and ant occupancy that were rarely observed. 
For example, the no-partner scenario modeled a hypothetically ant-free cactus population, even though no such population exists to our knowledge, by applying the statistical knowledge gleaned from vacant plants across the size distribution. 

In all scenarios that included any ant partners, we preserved the observed pattern of size-dependent vacancy/occupancy (estimated through the ant transition sub-model) and manipulated partner identity conditional on occupancy. 
This means, for example, that the \textit{C. opuntiae}-only scenario included two possible states, vacancy and occupied by \textit{C. opuntiae}. 
While our statistical models allow us to extrapolate the demographic performance of ant-tended plants to small sizes that are typically vacant, the natural history of this system tells us that this is not biologically sensible. 
Small, non-reproductive plants are typically vacant because they do not produce extrafloral nectar, and once plants begin producing nectar they are nearly always ant-tended \citep{Miller2014}. 
Our simulation experiments preserved this basic biology, avoiding tiny ant-occupied plants that do not and could not occur in nature. 

The partner diversity treatment scenarios required additional assumptions about the mechanisms that give rise to observed occupancy patterns. 
Based on evidence that EFN-bearing cacti are nearly always ant-occupied \citep{Miller2014}, we assume that ant partners competitively exclude one another from EFN-bearing cacti and that competition is zero-sum. 
This means that, in scenarios that remove species from the partner community, remaining species gain access to plants that the removed species would have tended. 
In Appendix \ref{appendix}, we present results under an alternative assumption, that ant visitation is limited by factors other than availability of cactus EFN (e.g., nesting sites or off-plant dietary resources), such that when a species is removed from the partner community, the plants it would have occupied remain vacant. 

\subsubsection*{Temporal stochasticity experiments}
Under the portfolio effect hypothesis, partner diversity may confer a fitness advantage when the benefits of alternative partners are not perfectly synchronized across temporal environmental variation, yielding an advantage of a diverse ``portfolio'' of partners when the environment fluctuates. 
Our statistical estimation of ant-specific year random effects in the vital rates allows for this possibility. 
We constructed two versions of the stochastic, multi-state IPMs that allowed us to test this hypothesis by exploring two scenarios of environmental variation.
The 'non-synchronous' (NS) version included ants effects that varied uniquely across time and the 'synchronous' (S) version included  ant effects that were forced to be the same across species. 
First, we evaluated the model using empirical estimates for the $\pmb{\tau}_{a}$ vectors that describe ant-specific year deviates. 
In this scenario, good years and bad years can differ between ant states, according to the empirical parameter estimates. 
We also quantified from the fitted random effects how tightly inter-annual variation was correlated between ant states.
Second, we averaged the ant-dependent vital rates (survival, growth, flowerbud viability) across ant-specific year random effects, thus ensuring that plants in all ant states fluctuated synchronously in response to temporal environmental variation. 
We evaluated a second, ``synchronized'' version of the model that effectively turns off any portfolio effect, holding all else equal. 
Both scenarios of temporal stochasticity, non-synchronized and synchronized, were run for all eight ant partner scenarios described above. 

\subsubsection*{Statistical inference on fitness consequences of partner identity and diversity}
The range of models we created generated many outputs; we focus our inference on the following specific contrasts. 
First, to determine whether ant occupancy and partner diversity are beneficial, we calculated a posterior distribution of $\lambda_{NS}$ for each of four partner richness levels (zero, one, two, three), averaging over composition scenarios within each level. 
If cactus fitness increases with partner richness, this would be interpreted as evidence for benefits of partner diversity. 
Second, to determine whether each partner, in isolation, confers a fitness advantage and to rank alternative partners, we contrasted the fitness of each single partner scenario (\textit{C. opuntiae} only, \textit{L. apiculatum} only, Other only) against vacancy (zero partners). 
Third, to determine whether any benefits of diversity are due to the sampling effect or complementarity, we contrasted the fitness of multi-partner scenarios against the single best partner scenario. 
If the best multi-partner scenario exceeds the fitness associated with the best single partner, this would be interpreted as evidence of complementarity, a true benefit of diversity \emph{per se}. 
Alternatively, the sampling effect hypothesis predicts that no multi-partner scenario yields higher plant fitness then the best single partner. 
It is also possible that multi-partner scenarios yield lower fitness than the single best partner, which would be consistent with an opportunity cost of diversity. 
Fourth, to quantify any contribution of the portfolio effect, we contrasted $\lambda_{NS}$ of the full (four-state) scenario to vacancy for synchronized and non-synchronized responses to temporal stochasticity (as a measure of how much benefit is recieved from all partners being present). 
If the portfolio effect confers a benefit of diversity, the fitness advantage of having all vs. no partners should be greater when temporal fluctuations are not synchronized across ant states.

We base our statistical inferences on the posterior probability distributions of the contrasts described above. 
For example, the contrast of \textit{C. opuntiae} with vacancy yields a posterior distribution of the difference in $\lambda_{NS}$ ($\Delta\lambda_{C-V}$). 
We can quantify from this distribution our certainty in the mutualistic effect of \textit{C. opuntiae}, given the data, as $Pr(\Delta\lambda_{C-V}>0)$. 
We apply similar logic to other contrasts described above. 

 
\section*{Results}
\subsection*{What are the demographic effects of association with alternative parnters and how do these effects fluctuate across years?}
Over the 20-year data set, we found that ant partners influenced demographic performance of cactus hosts, and different ant partners had contrasting demographic effects across host vital rates. 
Plants tended by \textit{C. opuntiae} had a growth advantage, particularly for small plants, while plants in states of \textit{L. apiculatum}, Other ants, and vacancy had indistinguishable growth trajectories (Figure \ref{fig:Grow}).
For all ant states, growth was left-skewed, with transitions to sizes below the mean were more common than sizes above the mean. 
Similarly, for plants which were large enough to have ant visitors (larger than $0.8$ $log(cm^3)$), visitation enhanced cactus survival (Figure \ref{fig:Surv}).
Plants that were smaller than $-2$ $log(cm^3)$ (indicated by the dashed lines for ant estimates, Figure \ref{fig:Surv}) were estimated to experience survival boosts from vacancy.
Due to the assumptions in our models based on real ant presence data, tended plants of the size where vacancy would offer a boost are never seen and therefore do not influence the fitness estimates of cacti. \ali{}{I'm not 100\% certain on how to discuss this. I know that this still shows that crem offer a survival advantage, but I see that it could look like vacancy offers survival advantages. }
The smallest plants showed evidence that vacancy resulted in the highest survival rates, however these plants are never tended, meaning the models of ant-tended cacti survival before the solid lines are extrapolation.
Mean survival rates ranged from 7.7\% to 99.9\%, with the smallest plants the most vulnerable to mortality. 
\textit{C. opuntiae}-occupied plants had a survival advantage over other ant-tended plants, particularly at smaller sizes, consistent with the positive effects on growth. 
At larger sizes, plants in any state of ant occupancy had a survival advantage over vacant plants. 

\begin{figure}
	\includegraphics[width = 0.95\linewidth]{Figures/grow_contour_v2.png}
	\caption{This figure shows the next predicted size of cholla based on previous size with each individual ant partner. The solid colored lines (seen in all panels) are the next mean predicted size of cholla. The points (seen in panels a-d) are the observed data which informs these estimates. The black countour lines (seen in a-d) appear at 5\% increments showing where 5\%, 10\%, etc. of the data is expected to fall. They grey dashed line (in panel e only) shows the line where the next predicted size is the same as the previous (aka there is no growth on this line and below this line is shrinkage). }
	\label{fig:Grow}
\end{figure}

\begin{figure}[H]
\includegraphics[width=0.95\linewidth]{Figures/survival_plot.png}
\caption{This figure shows the estimated survival rates based on the size of the cactus with each individual ant partner. The solid colored lines (shown on all panels) indicate the mean estimated survival rates. The dashed lines (shown in panel e) indicate extrapolations beyond existing data (where we estimated survival for plants tended by ants where we had never seen a tended cactus of that size). The grey area around the solid lines (shown in panels a-d) show the 90\% confidence interval for the estimates. The colored dots are the real data binned by size to show how our estimates align with real survival observations. A larger circle means we had more data on survival of plants of this size with this partner.}
\label{fig:Surv}
\end{figure}

We found evidence that ant visitation by \textit{L. apiculatum} ants leads to increased floral viability rates and that ant identity can influence the strength of viability benefits.
We observed mean viability rates of flowers between 40\% and 92\% (Figure \ref{fig:Viab}).
Ant partners influence the mean viability rate of flowers, with \textit{L. apiculatum}-tended plants experiencing the highest mean viability rate (86\%), followed by vacant and Other tended plants (at 60-61\%), and \textit{C. opuntiae} tended plants had the lowest floral viability rate (57\%).


\begin{figure}[H]
	\includegraphics[width=0.95\linewidth]{Figures/Viab_v2.png}
	\caption{This figure shows the estimated distributions of floral viability rates compared to observed distributions of floral viability rates of cholla based on ant partner identity. The solid lines indicate the estimated viability distribution. The colored histograms represent the observed viability rates of plants with that partner. }
	\label{fig:Viab}
\end{figure}

We analyzed all other vital rates mentioned in the methods as well.
The results of each of these as well as the posterior predictive checks are all included in the Apendix \ref{appendix:A} : Figures \ref{app:AppA_Seeds_Per_Fruit} -- \ref{app:AppA_Pre_Census_Surv}.

We analyzed the correlation coefficients of all models which included ant state as a predictor and found that the annually varying effects of each ant on growth were the most correlated (correlation coefficients averaged 0.63), and that the effects on survival were the least correlated (correlation coefficients averaged 0.36). 
High correlation indicates synchronicity, which is necessary for portfolio effect to occur because a central thesis of this effect is that partners must react differently to temporal environmental stochasticity.
For the growth model, annually varying randomn effects of Other ants were the least synchronous (correlation coefficients ranging from 0.28 to 0.39) while the effects of vacancy were the most synchronous (correlation coefficients ranging from 0.39 to 0.81).
For the viability model, annually varying random effects of \textit{L. apiculatum} ants were the least synchronous (correlation coefficients ranging from 0.06 to 0.44) while the effects of vacancy were the most synchronous (correlation coefficients ranging from 0.44 to 0.62).
For the survival model, annually varying random effects of \textit{C. opuntiae} ants were the least synchronous (correlation coefficients ranging from 0 to 0.15) while the effects of vacancy were the most synchronous (correlation coefficients ranging from 0.13 to 2.6).
This variety of synchronicity across ant states and vital rates indicates there is potential for portfolio effect as many of the ants effects revealed low synchronicity (low correlation of annual effects, indicating that ant identity impacts annual random effects), particularly in the survival model. 


\begin{figure}[H]
	\includegraphics[width=0.95\linewidth]{Figures/ant_RFX.png}
	\caption{This figure shows the mean affect of each ant partner on a) the estimated next size, b) the estimated survival, and c) the floral viability of cacti across every year of our study. These values are estimated from the fitted random effects of ant and year in our models. Each point represents the mean of the random effect of the identified model, ant, and year (e.g. the lowest dot in panel b) represents the mean effect of vacancy on survival rates in year 2011).}
	\label{fig:Annual_Ant}
\end{figure}
\ali{}{Honestly not sure if I should include an image for this one or just report some values? We should discuss. }

\subsection*{What are the frequency and direction of partner turnover across the plant life cycle?}
We found that 55\% of individual plants surveyed in the long-term data experienced at least one ant state transition on average, with very distinct size-dependence and directional patterns (Figure \ref{fig:Ant_Transition}). 
Vacancy is the most likely ant state of small plants ($\leq 10 log(cm^3)$).
Even when small plants are ant-tended at the start of the transition year, they are most likely to transition back to vacancy (Figure \ref{fig:Ant_Transition}b-d). 
The probability of becoming ant-tended increases with size, though it is not equally likely to be tended by all partners. 
For large plants that are initially vacant or tended by \textit{L. apiculatum} or Other ants, \textit{L. apiculatum} is the most likely next partner, suggesting that this partner species is able to colonize plants that were previously vacant or occupied by Other ants, and effectively retain plants that it previously occupied.  
\textit{C. opuntiae} were also able to retain plants they previously occupied, but not as well as \textit{L. apiculatum}: for plants that begin the transition year with \textit{C. opuntiae}, the probability that those plants remain occupied by \textit{C. opuntiae} at the end of the transition year is only slightly greater than the probability of take-over by \textit{L. apiculatum}, while take-over in the other direction is extremely rare. 
It is also notable that transitions away from the initial state of  \textit{L. apiculatum} were almost always transitions to vacancy (Figure \ref{fig:Ant_Transition}d), while transitions away from the initial states of \textit{C. opuntiae} and Other  were often transitions to other ants. 
This suggests a competitive hierarchy whereby \textit{L. apiculatum} may abandon low-value plants with litte nectar production but is almost never displaced from high-value plants. 

\begin{figure}[H]
	\includegraphics[width=0.95\linewidth]{Figures/transition.png}
	\caption{This figure shows the probability of being tended by each ant partner or vacant based on the size of the plant. Each panel shows these probabilities for a different previous ant state. The solid lines represent the mean probability of being tended by a specific partner. The colored points are the real data binned by size to show how our estimates align with real visitation observations. A larger circle means we had more data on visitation of plants of this size with this previous partner.}
	\label{fig:Ant_Transition}
\end{figure}

\subsection*{What is the net effect of partner diversity on plant fitness, and what mechanism(s) explain(s) this effect?}
By integrating vital rate results and ant transition dynamics into the multi-state IPM we can evaluate the fitness implications of different scenarios of partner diversity and identity. 
First, there was strong evidence that ant visitation had mutualistic fitness effects on plant partners. 
The lowest mean fitness was $\lambda_{NS,Vacant}$, the fitness of the cholla with no partners (Figure \ref{fig:LambdaMeans}b).
Across all 1+ partner scenarios, we found that the posterior $\lambda_{NS,Occupied}$ distributions were greater than $\lambda_{NS,Vacant}$ distributions 100\% of the time.
This indicates that ant visitation elevates fitness no matter the identity or number of partners.
Furthermore, we find no benefits of patner diversity here, with the fitness of the cholla with 1, 2, and 3 partners roughly equivalent (Figure \ref{fig:LambdaMeans} a).
From the means alonet appears that there is a reduction in fitness when all ants are present compared to 1 or 2 partners.
The contrasts of $\lambda_{NS}$ posterior distributions of 0 - 2 partner diversity scenarios to $\lambda_{NS}$ posterior distribution of 3 partners show that the overlap ranges from 58\% - 100\%, indicating high similarity between thr full partner diversity scenario and others.
This is highlighted in Figure \ref{fig:LambdaMeans} panel b.

\begin{figure}
	\includegraphics[width=0.91\linewidth]{Figures/Lambdas_Comp_lines.png}
	\caption{Panel a) shows the mean estimated $\lambda_{NS}$ and $\lambda_{S}$ for different numbers of partners (0-3) for the synchronous IPM (black circle) and the non-synchronous IPM (red diamond). Panels b-c) shows the mean estimated $\lambda_{NS}$ and $\lambda_{S}$ respectively, for each simulated combination of ant partner as the filled in circles. The lines show the posterior distribution spread of estimated $\lambda$ values. The letters in the legend correspond to what ant partners are present (V = Vacant, C = \textit{C. opuntiae}, L = \textit{L. apiculatum}, O = other). }
	\label{fig:LambdaMeans}
\end{figure}

We found that the difference in posterior distributions of any ant scenario ($\lambda_{NS,A} - \lambda_{NS,B}$, where A and B are arbitrary ant diversity scenarios that are not vacancy) were between 31\% and 83\% different.
The posterior distributions of $\lambda$ for each diversity scenario are visualized in Figure \ref{fig:LambdaMeans} b.
This high level of overlap between posterior distributions indicates that while each of these partners are beneficial, there does not appear to be a significant benefit of partner diversity within this system.

The lack of diversity benefits are not driven by the high overall frequency of \textit{L. apiculatum}. 
Using the simulations where all ants had equal frequencies across sizes (further explained and analyzed in Appendix C), we found the same fitness patterns as in the competitive exclusion model described above.
Equal probability for transitioning into any ant state meant that the numbers of \textit{C. opuntiae} and Other ants were boosted significantly.
Despite this, the fitness of scenarios including these less frequent ants were not increased meaningfully.

We found no evidence of portfolio effect.
The effect of all ant partners can be measured as $\lambda_{All} - \lambda_{Vacant}$ (Figure \ref{fig:Portfolio}).
We are are 100\% confident that when all ants are present the cholla experience higher fitness than when no ants are present according to both the synchronized (S) and non-synchronized (NS) model scenarios. 
When subtracting these two resulting vectors from each other (($\lambda_{NS,All} - \lambda_{NS,Vacant}$) - ($\lambda_{S,All} - \lambda_{S,Vacant}$)), we found that there is no real difference between the two scenarios, meaning we have no evidence of portfolio effect.

\begin{figure}
	\includegraphics[width=\linewidth]{Figures/portfolio_effect.png}
	\caption{This figure shows the distribution of $\lambda_{NS,All}-\lambda_{NS,Vacant}$ in pink and $\lambda_{S,All}-\lambda_{S,Vacant}$ in green. The vertical dashed line shows where the effect of partners on the fitness of the cholla is 0 (to the left the partners have a negative effect, to the right the partners are beneficial).}
	\label{fig:Portfolio}
\end{figure}



\section*{Discussion}
% mini abstract paragraph
Mutualisms commonly involve multiple partners but the ecological consequences of partner diversity remain poorly understood. 
Here we show that while alternative partners may be ecologically different, their fitness effects on a shared mutualist can be effectively equivalent and interchangeable.
The results of our heirarchical models revealed that different ant partners had different effects on vital rates, with \textit{C. opuniae} tended plants \tom{experiencing slight advantages in growth}{I don't think this is true if you look at my v1 figure.} and survival when small, and \textit{L. apiculatum}-tended plants experiencing floral viability advantages.
The results of our synchronous and non-synchronous stochastic IPM revealed that all diversity scenarios which included any partners resulted in the highest possible fitness for tree cholla, suggesting that while ant visitation is beneficial, partner identity and diversity are inconsequential in this system. 
These results highlight that partners can differ meaningfully while still resulting in the same fitness benefits for a focal mutualist.

% Connect to broader patterns in the lit
Previous studies have reported complementarity \citep{Palmer2010,Rezende2007,Afkhami2021} while we found that there are no benefits of diversity in our system.
Vital rate results indicated that different ant partners affected different vital rates uniquely (Figures \ref{fig:Grow},\ref{fig:Surv},\ref{fig:Viab}).
On their own, these results suggested a likely benefit of diversity through complementarity, such that each ant partner offered a unique form of herbivore protection across ontogeny and or across different life history processes. 
However, our multi-state IPM results, which synthesize all life history processes and accouns for partner identity and turnover, revealed an overall positive effect of ant visitation but no apparent benefit of diversity, raising questions about how to resolve the tension between these two sets of results. 

We speculate that our results are driven by the vital rate sensitivity structure of this population. 
Like other long-lived, iteroparous species \citep{franco2004comparative}, tree cholla fitness is most sensitive to the growth and survival of established individuals \citep{Miller2009,elderd2016quantifying}, which are virtually guaranteed many years of reproductive opportunities once they reach a size that is protected from mortality. 
Differences between alternative partners were most pronounced either in reproductive rates (Figure \ref{fig:Viab}), which contribute relatively weakly to fitness (because a long reproductive lifespan overrides individual reproductive bouts), or in survival at small sizes, where mortality risk was high (Figure \ref{fig:Surv}) and the probability of ant tending was low (Figure \ref{fig:Ant_Transition}). 
At larger sizes with low mortality risk, ant-tended plants had a modest survival advantage over vacant plants regardless of ant identity, and that result dominates integrated measures of fitness because of the high sensitvity of established plant survival. 
The very real demographic differences between partner species in other vital rates and at other sizes do not register nearly as strongly in the currency of fitness. 

% why we don't see the vital rate differences in real life
%Despite the apparent benefits for small \textit{C. opuntiae} tended cacti, these benefits are almost never experienced due to chronic vacancy of small cacti (Figure \ref{fig:Ant_Transition}).
%We estimated that cacti between the approximate sizes of -$5 - 5 log(cm^3)$($0.01 - 148 cm^3$) would have important advantages in growth and survival if tended by \textit{C. opuntiae} ants.
%At these sizes plants are very rarely tended ($< 10\%$ occupancy) due to lack of EFN production.
%This means, while our vital rate models predict benefits for small ants, the assumptions we put in place for ant tending in our IPM means that these benefits are extremely rare and therefore have next to no impact on the fitness of the cacti.
%This is not to say that growth and survival are not important vital rates, in fact they are likely to be the two most important in the model. 
%In the literature, growth and survival for established plants are the vital rates that fitness estimates are most sensitive to \cite{Eacker2017}.
%This indicates that the boosts recieved from ant tending of small plants would not be likely to make a large impact on fitness estimations even if this were allowed to occur in this model.

% Portfolio Effect
Our work explcitly incorporated temporal environmental stochasticity, which raises the opportunity for portfoilio effect as a mechanism of diversity benefits. 
Yet, we find no evidence of portfolio effect within our system. 
When partners exhibit different reactions to varying environments, interacting with multiple partners can lead to more consistent benefits across time \citep{Batstone2018}.
We did not directly measure the reactions of ant partners to temporal environmental stochasticity, rather we measure how the effects of each partner on cholla demography fluctuate through time, and whether those fluctuations are correlated.
The vital rate results led us to believe that portfolio effect may be at play due to the differences in ant responses to annual variation.
However, there is no stabilization of fitness across temporal heterogeneity and therefore no benefits of diversity due to the portfolio effect.
Other portfolio effect studies \citep{Lazaro2022,Tornos2023} found that asynchrony was linked to weak evidence of portfolio effect.
One study \cite{dallas2022temporal} found that while portfolio effect was easy to show in theoretical models, it is often very weak or nonexistent in empirical data across many systems. 
This indicates that it may be very difficult to detect, disguised by different mechanisms, or uncommon in nature.

% These results are robust to different assumptions about turnover patterns & what we don't know about turnover
Partner turnover is likely a significant driver behind the fitness we see within Cholla populations, however the processes which drive the actual turnover frequency and directions remain a mystery in this system.
In the literature, it is clear that the frequency of partner turnover can have big effects on the fitness of the focal mutualist \cite{Fiala1994, Horvitz1986, Oliveira1999, Sachs2004}.
The direction of partner turnover is also important when the identity of partners impacts the quality of benefits recieved \cite{Fonseca2003, Alonso1998, Dejean2008, Noe1994}.
In our system we found that there are distinct patterns to the directions and frequency of partner turnover, but we still don't understand the mechanisms driving these turnovers.
In the model presented in this paper we assumed that competitive exclusion was the driving mechanism behind the patterns we see.
This assumption would indicate that ants are competing for each individual cacti, likely based on EFN quality and composition \citep{Heil2004,Heil2010}, and that the more common an ant is the better a competitor it is.
In our simulations this means that when one ant is removed the ants which are left occupy the newly vacated cacti so that the proportion of vacant cacti never changes.
We included this assumption because we believe it to be the most biologically reasonable based on the personal observations of multiple researchers.
However, it is possible this assumption is incorrect, so we created two alternative simulations to determine how robust our results were to various assumptions about partner turnover drivers.
The first alternative we tested is a model that assumes all ants have the same competitive ability rather than the differences in competitive ability we assume cause observed variety in ant frequency.
In this hypothetical simulation all ants have equal probability of occupying any given cactus, but the level of vacancy across the population remains consistent with observed data.
This simulation was designed to test if the overwhelming frequency of \textit{L. apiculatum} may be overriding any potential advantages of other partners.
The results of this simulation (Appendix \ref{appendix:C}: Figure \ref{app:EqualLambdaMeans}) were effectively the same as the competitive exclusion model presented in the body of this paper (Figure: \ref{fig:LambdaMeans}).
The second alternative we tested was a model that assumed rather than competition driving ant occupations, it was the population size of each ant species alone.
In this hypothetical simulation when one ant species is removed from the model the cacti are left vacant.
We believe this is an unlikely scenario as there appear to be vast populations of the ant species in the area. \ali{}{This is just personal observation so it might be best to remove this?}
This simulation was designed to test if the fitness was more sensitive to changes in vacancy across the population than the frequency of any particular ant species.
The results of this simulation revealed that when vacancy increased sampling effect is at play (Appendix \ref{appendix:C}: Figure \ref{app:FreqLambdaMeans}).
This result highlights that the primary driver behind the results we presented are ant presence and that it is only in cases oh unrealistically high vacancy that there are benefits to partner diversity.
Further information about both of these models can be found in Appendix C.
Together, these results indicate that our assertions that the fitness of the cacti are more sensitive to changes in vacancy than partner identity are robust across multiple assumptions.


% Value of Long-term data
This paper shows the importance of long-term datasets in investigating species interactions and calls for further use of long-term data. 
Previously studies have analyzed how partner identity and partner turnover impact focal mutualist fitness \cite{Fonseca2003, Dejean2008, Noe1994, Barrett2015, Bruna2014, Trojelsgaard2015}.
Separate studies have analyzed how inter-annual variability impacts focal mutualists \cite{Alonso1998, Alarcon2008, Ollerton2006, Horvitz1990, Lazaro2022}.
The long term data set we used gave us the unique ability to consider the combined effects of partner identity, partner turnover, and temporal stochasticity.
By piecing together complete life cycle information from long-term data, we gain a more nuanced understanding of the fitness consequences of specific demographic effects. 
For example, our previous study suggested that \textit{C. opuntiae} has overall parasitic fitness effects because activity of this species within tree cholla flowers can deter pollinators and reduce seed set \citep{Ohm2014}. 
Yet, the more complete analysis presented here, which accounts for reduced seed set alongside other demographic advantages, indicates that this species is clearly a mutualist and nearly as strong a mutualist as \textit{L. apiculatum}. 

% Herbivory
Herbivory is an important driver of the fitness of cholla in this system, as herbivores directly impact the growth, survival, and reproductive efforts of the cacti \citep{Miller2009}.
We have not explicitly incorporated herbivory but rather assume that herbivory can be captured in the effects they have on growth, survival, and reproductive effort observations.
Analysis of our observational herbivore data shows that vacant plants experience elevated levels of herbivory compared to tended ones (Appendix \ref{appendix:B}: Figure \ref{app:herb}).
In the future, further studies on the direct impacts of herbivory would bolster the results reported here and expand our knowledge within this system.


% Limitations
As with any study, there are limitations to consider when interpreting the results reported here.
These results highlighted in the paper are based on observational data regarding ant effects on plant demography rather than experimental data, meaning we are able to determine correlations but not causation.
However, we have previously conducted experimental manupulations which revealed that ant presence has direct impacts on plant demography in our system \citep{Miller2007}.
This combination of observational data backed up by experimental results gives us greater confidence in our causational interpretations than if we had only observational data.
Specifically, we are confident in claiming that ant presence causes increased performance in cactus vital rates and the fitness signals we are detecting are directly related to ant partner presence.
One further expansion could include nectar analysis in conjuntion with ant interactions to look not only at the ant parnter impacts on the plant demography, but also at the ability of plants to attract specific partners through nectar composition shifts. 

% tie back paragraph.
This study highlights that while partners within a mutualistic guild can be ecologically different, they may still be interchangable to a focal mutualist.
The individual estimation of each vital rate and explicit inclusion of ant partners in this IPM allowed us to determine that partner diversity is not beneficial in this system and may highlight circumstances under which we would expect similar results in other systems.
The individual vital rate estimation approach allowed us to determine that while different ant partners did impact each vital rate differently, the largest differences were estimated to occur at sizes when ants would not be present in real life or in vital rates that do not have a significant impact on the overall fitnes.
Considering partner imapcts on vital rates across the entire life cycle of the plant allows more power in determining the importance of these differences.
Based on our results we would expect to see that partner diversity does not matter when partner impacts occur during low sensitivity life stages.


\section*{Acknowledgments}
We are grateful to the Sevilleta community (LTER and US Fish and Wildlife Service) for providing a stimulating research environment and logistical support. 
Jeremiah Dye identified ants and Massa Takahashi identified herbivores. 
We acknowledge the many students and technicians who have helped collect annual census data for this project, including M. Donald, J. Fowler, T. Jordan-Millet, J. Moutouama, C. Oxley, K. Schraeder, B. Scherick, A. Sears, M. Tucker, and J. Xiong.
Financial support for this work came from the Sevilleta LTER (NSF DEB-0217774)

%%%%%%%%%%%%%%%%%%%%%
% Statement of Authorship
%%%%%%%%%%%%%%%%%%%%%
% This section should also be commented out while your MS is undergoing
% double-blind review. The specifics should of course be adapted to
% your paper, but the paragraph below gives some hints of possible
% contributions.

\section*{Data and Code Availability}
The data that support the findings of this study are openly available as a \href{https://portal.edirepository.org/nis/mapbrowse?packageid=knb-lter-sev.323.1}{data package} on the Environmental Data Initiative website, package id: knb-lter-sev.323.1


\renewcommand{\theequation}{A\arabic{equation}}
% redefine the command that creates the equation number.
\renewcommand{\thetable}{A\arabic{table}}
\setcounter{equation}{0}  % reset counter 
\setcounter{figure}{0}
\setcounter{table}{0}

%%%%%%%%%%%%%%%%%%%%%
% Bibliography
%%%%%%%%%%%%%%%%%%%%%
% You can either type your references following the examples below, or
% compile your BiBTeX database and paste the contents of your .bbl file
% here. The amnatnat.bst style file should work for this---but please
% let us know if you run into any hitches with it!
%
% If you upload a .bib file with your submission, please upload the .bbl
% file as well; this will be required for typesetting.
%
% The list below includes sample journal articles, book chapters, and
% Dryad references.
\bibliographystyle{apalike}
\bibliography{References.bib}


\newpage{}

\section*{Tables}
\renewcommand{\thetable}{\arabic{table}}
\setcounter{table}{0}

\renewcommand{\thetable}{\arabic{table}}
\setcounter{table}{0}

% I am creating a table here to include all parameter estimations and descriptions
  \begin{table}[]
  \begin{tabular}{l|l|l}
    \textbf{Parameter} & \textbf{Median ($95\%$ CI)} & \textbf{Prior Distribution} \\
    \hline
    %% Growth Parameters
    growth xi intercept vacant $\beta_{01}^g$ & $-5.210899 (-5.686865, -5.491787)$ & sDE\\
    growth xi intercept other $\beta_{02}^g$ & $-5.8288 (-5.956217, 1.766021) $&asdf \\
    growth xi intercept \textit{C. opuntiae} $\beta_{03}^g$ & $-4.529523 (-6.0770390, 0.1222112)$ & asdf\\
    growth xi intercept \textit{L. apiculatum} $\beta_{04}^g$ & $-5.106802 (-5.4499944, 0.5453901)$ & asdf\\
    growth xi size dependent vacant $\beta_{11}^g$ & asdf&asdf \\
    growth xi size dependent other $\beta_{12}^g$ & asdf&asdf \\
    growth xi size dependent \textit{C. opuntiae} $\beta_{13}^g$ & asdf&asdf \\
    growth xi size dependent \textit{L. apiculatum} $\beta_{14}^g$ &sadf &asdf \\
    growth omega intercept $\omega_0^g$ & & \\
    growth omega size dependent $\omega_1^g$ & & \\
    growth alpha intercept $\alpha_0^g$ & & \\
    growth alpha size dependent $\alpha_1^g$ & & \\
    \hline
    %% Germination Parameters
    1-year germination intercept $\alpha^{\gamma_1}$ & & \\
    2-year germination intercept $\alpha^{\gamma_2}$ & & \\
    \hline
    %% Survival Parameters
    survival intercept vacant $\beta_{01}^s$ & & \\
    survival intercept other $\beta_{02}^s$ & & \\
    survival intercept \textit{C.opuntiae} $\beta_{03}^s$ & & \\
    survival intercept \textit{L. apiculatum} $\beta_{04}^s$ & & \\
    survival size dependent vacant $\beta_{11}^s$ & & \\
    survival size dependent other $\beta_{12}^s$ & & \\
    survival size dependent \textit{C. opuntiae} $\beta_{13}^s$ & & \\
    survival size dependent \textit{L. apiculatum} $\beta_{14}^s$ & & \\
    \hline
    %% Probability of Flowering Parameters
    flowering intercept $\beta_0^f$ & & \\
    flowering size dependent $\beta_1^f$ & & \\
    \hline
    %% Floral Viability Parameters
    viability intercept vacant $\beta_01^v$ & & \\
    viability intercept other $\beta02^v$ & & \\
    viability intercept \textit{C. opuntiae} $\beta_03^v$ & & \\
    viability intercept \textit{L. apiculatum} $\beta_04^v$ & & 

  \end{tabular}
  \caption{This table includes the median estimates, the 95$\%$ confidence intervals, and the prior distribution for each parameter in each model.}
  \label{tab:Params}
  \end{table}

\section*{Figure legends}


\appendix
\label{appendix}
\section*{Appendix A: Additional Methods and Parameters} \label{appendix:A}
In addition to the models described in the body of the paper, we fit several other models using data from previous studies.
These models are described below.

\paragraph{Seeds Per Fruit}
With data from \cite{Miller2006}, we fit a model for the number of seeds produced by every fruit on a cholla ($\kappa(a')$) in year $t+1$ based on the ant partner $a'$ in year $t+1$.
We fit this model to seed data $y^{\kappa}$ using a Negative Binomial distribution and the log link function: 
$$y^{\kappa} \sim  Negative Binomial(\hat{\kappa},\hat{\phi})$$
%$$ \hat{\kappa } = \beta_{0}^{\kappa} \times a'$$
$$\hat{\phi} = \beta_{0}^{\phi}$$
The data used for this model did not include data on ants in the ``other" category, so we used the data from vacant plants to parameterize seeds per flower for plants with ``other" ants in the IPM.

We found that vacant plants produced the most mean seeds (147.2 per fruit), followed by \textit{L. apiculatum} tended plants (142.4 per fruit), and finally, \textit{C. opuntiae} tended plants (115.0 per fruit) (Figure \ref{app:AppA_Seeds_Per_Fruit}).

\begin{figure}
	\includegraphics[width=0.91\linewidth]{Figures/Seeds_Per_Fruit.png}
	\caption{Shows the distribution of seeds per fruit on plants tended by \textit{C. opuntiae} (purple), \textit{L. apiculatum} (Teal), and Vacant plants (Yellow). }
	\label{app:AppA_Seeds_Per_Fruit}
\end{figure}

\paragraph{Recruit Size Distribution}
%The recruit size model ($n(x',a')$) estimates the size distribution of all recruits } from a given year $t+1$, with no fixed or random effects. 
We fit this model to recruit size data $y^{\eta}$ using a Normal distribution with the identity link function: 
$$y^{\eta} ~\sim N(\hat{\eta},\hat{\sigma})$$
%$$\hat{\eta} = \beta_{0}^{\eta}$$
where $\hat{\sigma}$ is estimated with a non-informative prior. 

We found that the mean size of recruits is $log(-2.097) m^3$ with an interquartile range from $log(-2.173) m^3$ to $log(-1.712) m^3$ (Figure \ref{app:AppA_Recruit_Dist}).

\begin{figure}
	\includegraphics[width=0.91\linewidth]{Figures/rec_size.png}
	\caption{Shows the distribution of recruit sizes. The mean  size of a recruit is marked by the dark line in the green box, the green box shows the interquartile range of recruit sizes, and the dashed lines show the minimum and maximum of estimated recruit sizes. The points beyond these dashed lines are considered outliers. }
	\label{app:AppA_Recruit_Dist}
\end{figure}

\paragraph{Germination}
With germination data \cite{Miller2007}, we fit two models for the probability of germinating from the first year seedbank ($\gamma_1$) or the second year seedbank ($\gamma_2$) in year $t+1$, with no fixed or random effects.
These models were fit to germination data $y^{\gamma_1}, y^{\gamma_2}$  using the binomial distribution with logit link functions:
$$y^{\gamma_1} \sim Binomial(\hat{\gamma_1})$$
$$y^{\gamma_2} \sim Binomial(\hat{\gamma_2})$$
%$$logit(\hat{\gamma_1}) = \beta_{0}^{\gamma_1}$$
%$$logit(\hat{\gamma_2}) = \beta_{0}^{\gamma_2}$$

We found that the mean germination rates for seeds in the seedbank for one year  is 0\%, with an interquartile range of 0\% and 1\%.
We found that the mean germination rates for seeds in the seedbank for a second year is 0\%, with an interquartile range of 0\% to 0.4\% (Figure \ref{app:AppA_Germ}).
Seeds are more likely to germinate in their first year in the seedbank, but most seeds will never germinate. 

\begin{figure}
	\includegraphics[width=0.91\linewidth]{Figures/germination.png}
	\caption{Shows the probability of a plant germinating during its first year in the seedbank (left) or during the second year in the seedbank (right). The mean germination rates are shown by the dark line near the bottom of the green box. The green boxes show the interquartile ranges of the germination rates for year 1 and 2 plants. The dashed line shows the maximum germination rates observed. The points above that show the outliers of our data. }
	\label{app:AppA_Germ}
\end{figure}

\paragraph{Pre-Census Survival}
With recruit census data \cite{Miller2006}, we fit a model for the probability of a seedling (which germinates in early Fall) surviving to when we census in May ($\delta$) of year $t+1$ (accounting for missed mortality events), with fixed effects of the previous size $x$ and random effects of the transect $m$.
We fit this model to pre-census survival data $y^{\delta}$ using a Bernoulli distribution with a logit link function: 
$$y^{\delta} ~ Bern(\hat{\delta})$$
%$$logit(\hat{\delta}) = \beta_{0}^{\delta} + m$$
where $m \sim N(0, \sigma_{transect}^2)$ is the random effect of transect where the recruited individual was analyzed for survival.

We found that plants have a 16.2\% probability of surviving from germination to the next census.
Our model estimated this very well, expecting a 16.3\% probability (Figure \ref{app:AppA_Pre_Census_Surv}).

\begin{figure}
	\includegraphics[width=0.91\linewidth]{Figures/seed_surv.png}
	\caption{Shows the distribution of the probability of plants which germinated between censuses surviving to the next census to be counted. The green line shows the predicted pre-census survival and the grey dashed line shows the actual observed pre-census survival.  }
	\label{app:AppA_Pre_Census_Surv}
\end{figure}

\section*{Appendix B: Observed Herbivory Data} \label{appendix:B}
Herbivory is an important driver in this population and shapes the range and demography of cholla.
Herbivore presence has been shown to negatively impact growth and fecundity of cholla populations \citep{Miller2009}.
Ant visitors are believed to offer defensive benefits to the plants they tend in this system, leading to the hypothesis that ant presence would be correlated with reduced herbivory.
Herbivory data was collected during censuses any time herbivores were identified on a plant. 
This involved noting the type and quantity of herbivores observed. 
This data has been taken consistently since 2017, so the analysis below considers 6 years of data.
We considered only plants which were reproducing, as they were likely to produce the highest quantity of EFN \citep{Miller2014}.
The proportion of reproducing plants that experienced herbivory was calculated for each ant state separately.
Analysis showed that ant presence is correlated with lower herbivore visitation.
40\% of vacant cacti experienced herbivory.
Plants tended by Other ants experienced similar, though lower, levels of herbivory on reproducing plants, with herbivores detected on 37.5\% of plants.
Herbivores were detected on 25\% of plants tended by \textit{C. opuntiae} ants and on 11\% of plants tended by \textit{L. apiculatum} ants.
These results indicate that ant presence is correlated with lower levels of herbivory and that partner identity has an impact on the level of herbivory.
They also indicate that the partner correlated with the lowest levels of herbivory is \textit{L. apiculatum} ants.
These findings are consistent with literature findings which show that \textit{L. apiculatum} ants are the most aggressive (therefore the most effective against herbivores), but differ from previous findings that \textit{C. opuntiae} may not offer defensive benefits \citep{Miller2007}.


\begin{figure}
	\includegraphics[width=0.91\linewidth]{Figures/herb_all.png}
	\caption{Shows the proportion of reproducing plants which are visited by herbivores. Each bar represents the subset of the cacti population in a different ant state.  }
	\label{app:herb}
\end{figure}

\section*{Appendix C: Alternative Ant Transition Simulations}\label{appendix:C}
In addition to the competitive exclusion model defined and analyzed in the body of the paper, we simulated results from several other potential models. 
We chose to include competitive eclusion as our primary results in the paper because we believe it to be the most biologically realistic.
However, in building and testing of alternative models we found that the method of ants occupying plants significantly impacts the fitness of the population. 
We tested two alternative transition models, one called the frequency based model and one called the equal likelihood model. 

\paragraph{Frequency Based Model}
The first alternative hypothesis we tested was what we called the frequncy based model.
In this model rather than the proportion of vacant cacti being maintained, the proportion of cacti occupied by each species is maintained and when one is removed it is replaced with vacancy.
This version of the model assumes that the frequency of each ant we see is reflective of the real frequency of populations rather than some other mechanism.
With this model we found very clear evidence of Sampling Effect in the system. 
When only  \textit{C. opuntiae}, Other ants, or both ants are present, there is very little difference in the fitness of the cacti from when no partners are present. 
Only when \textit{L. apiculatum} ants are present do we see an increase in the fitness of the focal mutualist (Figure \ref{fig:FreqLambdaMeans}a).
In this simulation, the more partners that are present the higher the fitness of the focal mutualist is, confirming that partner diversity would be beneficial through sampling effect if this transition model were correct.  (Figure \ref{fig:FreqLambdaMeans}b).

\begin{figure}
\includegraphics[width=0.91\linewidth]{Figures/Lambdas_Freq_lines.png}
	\caption{Panel a) shows the mean estimated $\lambda_{NS}$ and $\lambda_{S}$ for different numbers of partners (0-3) for the synchronous IPM (black circle) and the non-synchronous IPM (red diamond). Panels b-c) shows the mean estimated $\lambda_{NS}$ and $\lambda_{S}$ respectively, for each simulated combination of ant partner as the filled in circles. The lines show the posterior distribution spread of estimated $\lambda$ values. The letters in the legend correspond to what ant partners are present (V = Vacant, C = \textit{C. opuntiae}, L = \textit{L. apiculatum}, O = other). }
\label{app:FreqLambdaMeans}
\end{figure}

\paragraph{Equal Likelihood Model}
The second alternative hypothesis we tested was what we called the equal likelihood model.
In this model we preserved the observed pattern of size-dependent vacancy/occupancy, but occupancy was manipulated to be equally likely for all partner identities. 
This was designed to remove the effect overwhelming numbers of \textit{L. apiculatum} ants may have. 
Despite very different proportions, we found very similar outcomes to the competitive exclusion model analyzed in the paper. 
All ants are beneficial, but having more than one is not necessarily any better than having an individual species as a partner (Figure \ref{fig:EqualLambdaMeans}b).
Partner presence is beneficial, but neither identity nor number of partners appears to be important (Figure \ref{fig:EqualLambdaMeans}a).

\begin{figure}
\includegraphics[width=0.91\linewidth]{Figures/Lambdas_equal_lines.png}
	\caption{Panel a) shows the mean estimated $\lambda_{NS}$ and $\lambda_{S}$ for different numbers of partners (0-3) for the synchronous IPM (black circle) and the non-synchronous IPM (red diamond). Panels b-c) shows the mean estimated $\lambda_{NS}$ and $\lambda_{S}$ respectively, for each simulated combination of ant partner as the filled in circles. The lines show the posterior distribution spread of estimated $\lambda$ values. The letters in the legend correspond to what ant partners are present (V = Vacant, C = \textit{C. opuntiae}, L = \textit{L. apiculatum}, O = other). }
\label{app:EqualLambdaMeans}
\end{figure}


\section*{Appendix D: Posterior Checks and Model Validation}\label{appendix:D}
For each model fitted, we conducted two tests to determing if the fit was acceptable to use in our IPM. 
First, we checked the convergence of each parameter.
Below we show the convergence of all $\beta$ terms listed in the Statistical Modeling subsection of Methods.
Second, we checked the posterior fit, comparing the estimated values of each model to the $y$ values of the actual data.
We show these posterior checks below, split by ant partner where relevant.
%% Growth Figure
%% Survival Figure
\begin{figure}
\includegraphics[width = 0.45\linewidth]{Figures/surv_conv.png}
\includegraphics[width=0.45\linewidth]{Figures/surv_post.png}
\caption{The a) posterior convergence of the parameters estimated by the survival model and the b) posterior distribution of survival estimates (pink lines) for each ant species (1 = \textit{C. opuntiae}, 2 = \textit{L. apiculatum}, 3 = other, 4 = vacant) compared to the mean survival distribution (black line) of the real data.}
\label{fig:Surv_post}
\end{figure}
%% Reproduction Figure
\begin{figure}
\includegraphics[width = 0.45\linewidth]{Figures/repro_conv.png}
\includegraphics[width=0.45\linewidth]{Figures/repro_post.png}
\caption{The a) posterior convergence of the parameters estimated by the reproduction model and the b) posterior distribution of reproductive status estimates (pink lines) for each ant species (1 = \textit{C. opuntiae}, 2 = \textit{L. apiculatum}, 3 = other, 4 = vacant) compared to the mean reproductive status distribution (black line) of the real data.}
\label{fig:Repro_post}
\end{figure}
%% Flowering Figure
\begin{figure}
\includegraphics[width = 0.45\linewidth]{Figures/flow_conv.png}
\includegraphics[width=0.45\linewidth]{Figures/flow_post.png}
\caption{The a) posterior convergence of the parameters estimated by the number of flowers model and the b) posterior distribution of the number of flowers estimated (pink lines) compared to the mean distribution of observed flowers (black line).}
\label{fig:Flow_post}
\end{figure}
%% Viability Figure
\begin{figure}
\includegraphics[width = 0.45\linewidth]{Figures/viab_conv.png}
\includegraphics[width=0.45\linewidth]{Figures/viab_post.png}
\caption{The a) posterior convergence of the parameters estimated by the viability model and the b) posterior distributions of floral viability estimates (pink lines) for each ant species (1 = \textit{C. opuntiae}, 2 = \textit{L. apiculatum}, 3 = other, 4 = vacant) compared to the mean floral viability distribution (black line) of the real data.}
\label{fig:Viab_post}
\end{figure}
%% Seeds Per Fruit
\begin{figure}
\includegraphics[width = 0.45\linewidth]{Figures/seed_conv.png}
\includegraphics[width=0.45\linewidth]{Figures/seed_ant_post.png}
\caption{The a) posterior convergence of the parameters estimated by the seeds per fruit model and the b) posterior distributions of seeds per fruit estimates (pink lines) for each ant species (1 = \textit{C. opuntiae}, 2 = \textit{L. apiculatum}, 3 = vacant) compared to the mean seeds per fruit distribution (black line) of the real data.}
\label{fig:seed_post}
\end{figure}
%% Fruit survival 
% \begin{figure}
%% 	\includegraphics[width = 0.45\linewidth]{Figures/                .png}
%% 	\includegraphics[width=0.45\linewidth]{Figures/               .png}
% 	\caption{The a) posterior convergence of the parameters estimated by the fruit survival model and the b) posterior distributions of fruit survival estimates (pink lines) compared to the mean fruit survival distribution (black line) of the real data.}
% 	\label{fig:fruit_surv_post}
% \end{figure}
% %% Ant Transitions
% \begin{figure}
%% 	\includegraphics[width = 0.45\linewidth]{Figures/             .png}
%% 	\includegraphics[width=0.45\linewidth]{Figures/               .png}
% 	\caption{The a) posterior convergence of the parameters estimated by the ant transitions model and the b) posterior distributions of next ant partners estimates (pink lines) for each previous ant species (1 = \textit{C. opuntiae}, 2 = \textit{L. apiculatum}, 3 = other, 4 = vacant) compared to the mean next ant partner distribution (black line) of the real data.}
% 	\label{fig:Transitions_post}
% %% Germination
\begin{figure}
	\includegraphics[width = 0.45\linewidth]{Figures/germ1_conv.png}
	\includegraphics[width=0.45\linewidth]{Figures/germ2_conv.png}\\
	\includegraphics[width=0.45\linewidth]{Figures/germ1_post.png} 	\includegraphics[width=0.45\linewidth]{Figures/germ2_post.png}
	\caption{The a-b) posterior convergence of the parameters estimated by the germination from year one seedbank and germination from year two seedbank models respectively. The c-d) posterior distributions of floral viability estimates (pink lines) compared to the mean germination distribution (black line) of the real data for first year germinants and second year germinants respectively.}
	\label{fig:Germ_post}
\end{figure}

%% Pre-Census Survival
\begin{figure}
	\includegraphics[width = 0.45\linewidth]{Figures/seed_surv_conv.png}
	\includegraphics[width=0.45\linewidth]{Figures/seed_surv_post.png}
	\caption{The a) posterior convergence of the parameters estimated by the pre-census survival model and the b) posterior distribution of the pre-census survival estimated (pink lines) compared to the mean distribution of observed pre-census survival (black line).}
	\label{fig:Pre_Surv_post}
\end{figure}
%%

\subsection*{Statistical Models -- Results}
Below are the results reorted of all statistical models not described in the main body of the text. 

%% Reproduction Model
\paragraph{Reproduction Model}
The probability of a plant reproducing in a given year is highly size dependent. 
The mean probability of reproducing remains at about 0\% until the plant reaches a medium size, after which the mean probability of reproducing increases steadily before reaching about 100\% at large sizes. 



%% Seeds Produced
\paragraph{Seeds Per Flower Model}
Each viable flower on a plant produces between 97 and 257 seeds.
This number is affected by the ant partner present, as shown in previous work \citep{Ohm2014}. 
\textit{C. opuntiae} tended plants produce a mean of 115 seeds per flower. 
\textit{L. apiculatum} tended plants produce a mean of 143 seeds per flower.
Vacant plants produce a mean of 148 seeds per flower. 
Comparison between posterior distributions revealed that \textit{C. opuntiae} tended plants produced fewer seeds per flower than \textit{L. apiculatum} tended plants and vacant plants 80\% and 87\% of the time.
Vacant plants produced more seeds per flower than \textit{L. apiculatum} tended plants only 57\% of the time.
We are confident that \textit{C. opuntiae} tended plants produce the fewest seeds per flowers.

%% Precensus Survival\
	\paragraph{Precensus Survival Model}
	Pre-census seed survival rates fall between 0\% and 95\% with the mean pre-census seed survival at 18\%.
	
	%% Germination
	\paragraph{Germination Model}
	Seeds have a significantly higher probability of germinating in year one than in year two.
	Seeds in year one experience germination rates between 50\% and 100\% with a mean of 62\% germination.
	Seeds in year two experience germination rates between 50\% and 98\% with a mean of 58\% germination.
	
	
	%% Recruit size distribution
	New recruits are expected to be between the sizes of 0.11 $cm^3$ and 0.38 $cm^3$ with a mean size of 0.20 $cm^3$.



\end{document}
