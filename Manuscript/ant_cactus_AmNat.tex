\documentclass[11pt]{article}
\usepackage[sc]{mathpazo} %Like Palatino with extensive math support
\usepackage{fullpage}
\usepackage[authoryear,sectionbib,sort]{natbib}
\linespread{1.7}
\usepackage[utf8]{inputenc}
\usepackage{lineno}
\usepackage{titlesec}
\newcommand{\tom}[2]{{\color{red}{#1}}\footnote{\textit{\color{red}{#2}}}}
\newcommand{\ali}[2]{{\color{pink}{#1}}\footnote{\textit{\color{pink}{#2}}}}  


\titleformat{\section}[block]{\Large\bfseries\filcenter}{\thesection}{1em}{}
\titleformat{\subsection}[block]{\Large\itshape\filcenter}{\thesubsection}{1em}{}
\titleformat{\subsubsection}[block]{\large\itshape}{\thesubsubsection}{1em}{}
\titleformat{\paragraph}[runin]{\itshape}{\theparagraph}{1em}{}[.]\renewcommand{\refname}{Literature Cited}


%%%%%%%%%%%%%%%%%%%%%
% Line numbering
%%%%%%%%%%%%%%%%%%%%%
%
% Please use line numbering with your initial submission and
% subsequent revisions. After acceptance, please turn line numbering
% off by adding percent signs to the lines %\usepackage{lineno} and
% to %\linenumbers{} and %\modulolinenumbers[3] below.
%
% To avoid line numbering being thrown off around math environments,
% the math environments have to be wrapped using
% \begin{linenomath*} and \end{linenomath*}
%
% (Thanks to Vlastimil Krivan for pointing this out to us!)

\title{Thank you, next: partner turnover elevates benefits of mutualism for an ant-tended plant}

% This version of the LaTeX template was last updated on
% May 11, 2023.

%%%%%%%%%%%%%%%%%%%%%
% Authorship
%%%%%%%%%%%%%%%%%%%%%
% Please remove authorship information while your paper is under review,
% unless you wish to waive your anonymity under double-blind review. You
% will need to add this information back in to your final files after
% acceptance.

\author{Alexandra Campbell$^{1,\ast}$ \\ 
	Tom E.X. Miller$^{1}$}

\date{}

\begin{document}
	
	\maketitle
	
	\noindent{} 1. Program in Ecology and Evolutionary Biology, Department of BioSciences, Rice University, Houston, Texas 77005;
	
	\noindent{} $\ast$ Corresponding author; e-mail: amc49@rice.edu.
	
	
	\textit{Manuscript elements}: 
	
	\bigskip
	
	\textit{Keywords}: 
	
	\bigskip
	
	\textit{Manuscript type}: Article.
	
	\bigskip
	
	\noindent{\footnotesize Prepared using the suggested \LaTeX{} template for \textit{Am.\ Nat.}}
	
\linenumbers{}
\modulolinenumbers[3]

\newpage{}

\section*{Abstract}


\newpage{}

\section*{Introduction}

% The journal does not have numbered sections in the main portion of
% articles. Please refrain from using section references (à la
% section~\ref{section:CountingOwlEggs}), and refer to sections by name
% (e.g. section ``Counting Owl Eggs'').
Mutualisms are species interactions where all participants benefit, leading to higher individual fitness and increased population growth rates. 
They are among the most widespread species interactions \citep{Bronstein1994,Chamberlain2014,Frederickson2013}, but can deteriorate into commensalism or parasitism  \citep{Rodriguez-Rodriguez2017,Song2020,Mandyam2014,Thrall2007, Bahia2022}.
Mutualisms are considered more context dependent than other species interactions \citep{Chamberlain2014,Frederickson2013}, meaning the magnitude and sign of interaction strength are often determined by environmental conditions and species' identities.

Mutualism is defined at the level of a species pair (+/+) but these interactions are embedded within multi-species communities, and growing evidence suggests that pairwise interactions are poor predictors of the net effects of multi-species mutualism \citep{Afkhami2014,Palmer2010}. 
A focal mutualist may interact with multiple guilds of partner types (e.g., plants that interact with pollinators, seed dispersers, soil microbes, and ant defenders) or with multiple partner species within the same guild (e.g., plants visited by multiple pollinator species). 
Within a mutualist guild, partner species often differ in the amount or type of goods or services they provide, making partner identity an important source of contingency in mutualism \citep{Stanton2003}. 
Whether and how partner diversity modifies the demographic effects of mutualistic interactions remain open questions within relevance in applied settings \citep{rogers2014bee}. 

There are multiple mechanisms by which partner diversity can influence the net benefits accrued by a focal mutualist -- mirroring the mechanisms by which, at a larger scale of organization, biodiversity can influence ecosystem function (cite BEF chapter). 
First, when there is a hierarchy of fitness effects -- a consistent ranking of best to worst mutualists -- a more diverse sample of the partner community may be more likely to include the best partner \cite{Frederickson2013}.
This can lead to an apparent benefit of diversity driven by a sampling effect \cite{Batstone2018}. 
However, if partner associations are mutually exclusive then partner diversity may impose opportunity costs, leading to negative effects of a diverse mutualist assemblage relative to exclusive association with the single best partner \citep{Miller2007}. 
Second, even within a single mutualist guild, the benefits conferred by alternative partner species can vary in type, and not just degree \cite{Stachowicz2005,Bronstein2006,Stanton2003}. 
This can lead to a positive effect of partner diversity through complementarity of alternative functions \cite{Batstone2018}. 
Interference or synergies between partners can make their combined effect different than the expected from the sum of complementary functions (cite). 
Third, partner species can have species-specific responses to the environment, either spatially \citep{Ollerton2006} or temporally \citep{Alarcon2008}. 
Multiple partners can therefore act as a 'portfolio' that stabilizes fitness benefits across spatial or temporal heterogeneity, leading to positive effects of partner diversity through the portfolio effect \cite{Batstone2018,Lazaro2022}. 

Partner diversity can have different effects depending on whether partners are present all at once or sequentially (partner turnover) \citep{Djieto-Lordon2005, Ness2006, Bruna2014}. 
Sequential associations are likely when alternative partners engage in interference competition for access to a shared mutualist (cite examples, including non-ant-plant examples). 
Turnover can happen at different timescales, from minutes to years \citep{Oliveira1999,Horvitz1986}. 
The frequency of partner turnover can impact the level of benefits received by the focal mutualist, particularly if the benefits continue to accumulate (e.g., when sequential partners provide complementary functions) or if they saturate over time \citep{Sachs2004}.
Directionality of turnover can also influence diversity effects, particularly if partner identity changes consistently across ontogeny of a focal mutualist \citep{Fonseca2003}.
For example, plant susceptibility to enemies can change across life stages \citep{Boege2005,Barton2010}, so the benefits of defensive mutualism with ants are greatest when more defensive partner species align with more vulnerable life stages \citep{Djieto-Lordon2005}.

Defensive ant-plant mutualisms -- where plants provide food and/or housing to ants that in turn defend them from enemies -- are widespread interactions that offer valuable model systems for the ecology and evolution of mutualism \citep{Bronstein1998, Bronstein2006}. 
Extrafloral nectar (EFN) bearing plants can serve as dietary resources that promote ant abundance and colony size \citep{Byk2011, Ness2009, Ness2006}.
Presence of defensive ant partners is often linked to herbivory reduction \citep{Trager2010, Rudgers2010} and demographic advantages for the plant partner \citep{Baez2016}.
\tom{These interactions are almost entirely studied from the perspective of plant fitness \citep{Bronstein1994,Bronstein1998}, with little recorded about the impacts on ant fitness \citep{Lanan2013}.}{Here is a good example of something to avoid. You are highlighting a gap that your paper actually does not fill. Never do this in an Intro, sometimes it can be useful in a discussion as a future direction.}
Defensive ant-plant mutualisms are commonly multi-species, where a guild of ant partner species share, and often compete for, a plant mutualist \citep{Bronstein1998, Beattie1985, Trager2010, Agrawal1998}.
Ant partners can vary in their ability to deter herbivores \citep{bruna2004}, and visitation by low quality ant partners can prevent visitation by higher quality partners, consequently causing a reduction in fitness through missed opportunity costs \citep{Fraser2001, Frederickson2005}.
\tom{Another source of temporal variation is succeptibility to herbivory can also vary significantly throughout the life stages of the plant \citep{Boege2005}, suggesting that the order and timing of successive partners is important to the effectiveness of ant partners.
Temporal dynamics of partner visitation therefore have important impacts on the fitness of the plant partners in these interactions \citep{Barton2010, Boege2005, Fonseca2003}.}{I think you are mixing two ideas here: ontogenetic variation in ant-plant associations (eg directional turnover) and temporal stochasticity that is independent of plant ontogeny. Your study considers both and I think this needs to be set up more clearly.}
\tom{Recently many studies have investigated how partner diversity in these guilds has been shown to lead to either increased plant fitness \citep{Palmer2010, Afkhami2014} or decreased plant fitness (often in more highly specialized interactions) \citep{Barrett2015, Ushio2020}, stressing the importance of considering all ant partners as unique within these guilds. }{I think you need a more effective argument for why your study is necessary given that there are previous studies that have yielded results in both directions.}


This study examined the consequences of partner diversity in a food-for-protection mutualism between the tree cholla cactus (\textit{Cylindriopuntia imbricata}), a long-lived EFN-bearing plant, and multiple species of ant partners.
Previous studies have shown that herbivory by specialized insect herbivores negatively affects plant fitness \cite{Miller2009}, and ant defense reduces herbivore damage \cite{Miller2007}. 
Tree cholla are tended by two common ant species (\textit{Liometopum apiculatum} and \textit{Crematogaster opuntiae}) and several infrequent species, all of which are ground-nesting. 
These ant species locally co-occur at the scale of meters, but individual plants are typically tended by only one species that patrols the plant around-the-clock and maintains control of the plant's nectar resources for an entire growing season \citep{Ohm2014,donald2022does}. 
Switches between partner species, or between vacancy and ant occupancy, commonly occur from one growing season to the next \citep{Miller2007}. 
Prior experiments suggested a hierarchy of mutualist quality, with \textit{Liometopum apiculatum} providing strong fitness benefits and \textit{Crematogaster opuntiae} having net negative fitness effects because herbivore deterrence is outweighed by deterrence of pollinators \citep{Miller2007,Ohm2014}. 
However, those studies did not integrate the demographic effects of ant defense across the plant life cycle, nor did they account for inter-annual fluctuations in the benefits provided by alternative partners, and therefore may have missed important mechanisms through which different partner species, and their combination, may be beneficial. 

We used a unique long-term data set that allows us to explore mutualistic associations with multiple partner species and how the demographic effects of alternative partner species varied across plant size structure and nearly 20 years of inter-annual fluctuations. 
We used this observational data set, contextualized by previous experiments, to ask whether and through which mechanism(s) partner diversity affects the fitness benefits of ant visitation for the focal plant partner. 
Specifically, we asked:
\begin{enumerate}	
	\item{What are the demographic effects of association with alternative partners and how do these effects fluctuate across years?}
	\item{What are the frequency and direction of partner turnover across the plant life cycle?}	
	\item{What is the net effect of partner diversity on plant fitness, and what mechanism(s) explain(s) this effect?}
\end{enumerate}
We used a hierarchical Bayesian statistical approach to estimate demographic vital rates for hosts in different states of ant occupancy, and to quantify state-dependent partner turnover. 
We then used a stochastic, multi-state integral projection model (IPM) that combines diverse effects on vital rates and pathways of partner turnover to quantify effects of partner diversity on plant fitness. 

\section*{Methods}
\subsection*{Study System}
  
This study was conducted in the Los Pi$\tilde{n}$os mountains, a small mountain chain located on the Sevilleta National Wildlife Refuge, a Long-term Ecological Research site (SEV-LTER) in central New Mexico, USA.
This is an area characterized by steep, rocky slopes, and perennial vegetation including grasses (\textit{Bouteloua eriopoda} and \textit{B. gracilis}), yuccas, cacti, and junipers. 
Tree cholla cacti are common in high Chihuahuan desert habitats, with their native range spanning the southwestern USA \citep{Benson1982}. 
These arborescent plants produce cylindrical segments with large spines. 
In the growing season (May to August in New Mexico), the plants initiate new vegetative segments and flower buds at the ends of existing segments. 
While most plants produce new segments every season, only those which are reproductively mature produce flower buds. 
%Tree cholla generally reach at least 9 years of age before beginning to reproduce \citep{Ohm2014}.
Like other EFN-bearing cacti, tree cholla secrete nectar from specialized glands on young vegetative segments and flower buds \citep{Ness2006,Oliveira1999}. 
Flower buds produce more and higher-quality EFN than vegetative segments, making reproductive cholla valuable mutualist partners. 

Tree cholla EFN is harvested by various ant species. 
At SEV-LTER, cholla are visited primarily by two species of formicoid, ground-nesting ants, \textit{Crematogaster opuntiae} and \textit{Liometopum apiculatum}, as well as other rarer species, including \textit{Forelius pruinosus} and unidentified species of \textit{Aphaenogaster}, and a \textit{Camponotus}.
\textit{L. apiculatum} are the most frequent visitors with $25\% - 60\%$ of tree cholla tended by these ants, followed by \textit{C. opuntiae} visiting between \tom{$0\% - 20\%$}{Are there years with zero CREM?} of cacti \citep{Donald2022} depending on the year. \tom{Up to $80\%$ of cacti remain vacant in any given year. }{This is probably the extreme high end of vacancy. I would give the ranges as you did the others.}
These ants rarely co-occur on a plant, likely due to interspecific competition \citep{Miller2007}: staged introductions of \textit{C. opuntiae} to \textit{L. apiculatum}-tended plants, and vice versa, provoke aggressive responses by resident ants (A. Cambpell, \textit{personal observation}).
Each cholla is visited by a single ant species for the duration of a season, and the species of the visitors can change from one season to the next. 
At the beginning of the growing season, when EFN production begins, the ground-nesting ants will begin visiting tree cholla.
They will visit the cholla every day during the season around the clock, with the most acticity around sunrise or sunset \citep{Ohm2014}. 
Smaller cholla are less likely to be visited because they produce very little EFN, so larger cholla, especially flowering individuals, are generally more highly tended \citep{Miller2014}. 
In late August, the tree cholla stop producing EFN and the ants vacate until the next growing season. 

There are a variety of insect herbivores and seed predators which attack the cholla \citep{Mann1969}. 
An unidentified weevil of the genus \textit{Gerstaekeria} feeds on vegetative and reproductive structures and implants their larvae within the plant tissue for the winter. 
A cactus bug, \textit{Narnia pallidicornis}, (Hemiptera: Coreidae) feeds on all cholla parts with a preference for the reproductive structures \citep{Miller2006}.
A seed predator, \textit{Cahela ponderosella}, (Lepidoptera: Pyralidae) attacks developing fruits pre-dispersal and oviposits in open flowers mid-growing season where larvae burrow into the ripening ovary. 
These predators can have significant negative impacts on the fitness of individual cholla and depress population growth \citep{Miller2009}.
There is experimental evidence that tree cholla tended by \textit{L. apiculatum} and \textit{C. opuntiae} experience less herbivory than plants from which ants were excluded \citep{Miller2007}. 

\subsection*{Data}
\tom{}{Give this section a more informative title.}	
The \tom{data collected are from a long-term dataset}{This grammar does not really make sense.} spanning 2004 to 2023 \tom{taken from 30 $\times$ 30 meter plots}{These plots only started in 2009.} at SEV-LTER. 
The data initially included 134 naturally occurring plants across 4 spatial blocks censused annually from 2004 to 2008.
Six of the plots were established in 2009 by tagging all existing plants within a 30 $\times$ 30 meter area. 
The final two plots were added to this census from 2011 onwards. 
Annually, in May we surveyed all individuals in these plots, taking demographic and partner data. 
For each cholla, we recorded plant survival from the last survey to the current survey. 
For surviving plants, we recorded the height (cm), maximum crown width (cm), and crown width perpendicular to the maximum (cm), which are used to calculate plant volume ($cm^3$) based on the \tom{volume of a cone with the mean of maximum crown width and perpendicular crown width as the diameter}{Check your volume function but I think it is actually the volume of an elliptical cone.}. 
We recorded the total number of flower buds, including how many were aborted and how many were not. 
\tom{We recorded all ant species present and the number of ants we could count in 30 seconds.}{Need to describe how plants were assigned to an ant state in the case of multiple ant species (rare, I know.)} 
We also recorded the numbers and species of herbivores on each plant.
\tom{}{Also need to describe recruit data, seed banks, seed counts per fruit.}
		
\subsection*{Statistical Modeling}
	
With the data described above we fit a series of generalized linear mixed models (GLMMs) in a hierarchical Bayesian framework to serve as vital rate sub-models of the IPM.
Many of the vital rates are estimated as a function of plant size, ant partner, or both.
\tom{Ant partner type is included as a predictor only where there are biological pathways through which ants could impact the outcome of that process. }{This needs to be explained.}
The biological sources of variance (including \tom{individual}{But you do not have an individual random effect.}, spatial, and annual variance) are accounted for by including year-to-year and plot-to-plot random effects in the models. 
Unless otherwise mentioned, all models use vague priors. 

\tom{The growth model ($G_j(y,x)$) estimates the size of cholla, with fixed effects of the previous size and ant partner and random effects of plot and year, using a Skew Normal distribution, with $\omega$ and $\alpha$ varying with the previous size. }{I think you need to explain this more thoroughly, and also explain why you used a Skewed Normal. Omega and alpha need to be defined and explained, and this is best done showing the full notation of the model.}
Ants are included as a predictor here because ant partners defend plants from herbivory, therefore decreasing the likelihood of segment loss.
The survival model ($S_j(x)$) estimates the probability of survival, with fixed effects of the previous size of the cholla and ant partner and random effects of plot and year, using a Bernoulli distribution. 
Ants are included as predictors here because ant partners defend cholla from herbivores and predators, decreasing the likelihood of mortality due to either of these. 
The reproduction model ($P(x)$) estimates the probability of reproducing each year, with fixed effects for the size and random effects of plot and year, using a Bernoulli distribution. 
The total flowers model ($F(x)$) estimates the total flowers produced by a plant, with fixed effects of size and random effects of plot and year, using a Negative Binomial distribution. 
The viability model ($V_i(x)$) estimates the proportion of flowers produced by a plant which are viable (not aborted), with fixed effects of the ant partner of the cactus and random effects of plot and year, using a Binomial distribution.
Ants are included as predictors here because they defend the cacti from seed predation which can lead to floral abortion. 
The ant transition rates model ($\tau_{i,j}(x)$) estimates the probability of a cactus being visited by an ant partner, with fixed effects of the previous size of the cholla and the previous ant partner and random effects of plot and year, using a Multinomial distribution.  
Ant partners are included as predictors here because partners may choose to return to the same cholla repeatedly or choose new ones, therefore the previous partner may be a good indicator of the next partner. 
The recruit size model ($n_j(x)$) estimates the size distribution of all recruits from a given year, with no fixed or random effects, using a Normal distribution. 
With germination data from Miller et al., 2009, we fit two Bayesian generalized linear models for the probability of germinating from a seed in the first year ($\gamma_1$) or the second year ($1 - \gamma_1$), with no fixed or random effects, using a Binomial distribution.
With data collected in a 2005-2006 recruit census, we fit a Bayesian generalized linear model for the probability of a seedling surviving to May ($\delta$) (accounting for missed mortality events), with fixed effects of the previous size and random effects of the transect, using a Bernoulli distribution. 
With data from Miller 2007, we fit a Bayesian generalized linear model for the number of seeds produced by every flower on a cholla ($\kappa$) based on the ant partner, using a Negative Binomial distribution. 
Ant partners are included as predictors here because they reduce floral abortion rates and therefore may lead to higher numbers of seeds. 
\tom{}{General comment here is that each of these models needs to be described in greater detail and shown in reproducible notation. You also need to describe the idea of transitions in size x,y and ant state i,j since this sets the stage for the structure of the IPM.}

To obtain posterior estimates of the demographic parameters, we fit models using Markov chain Monte Carlo (MCMC) simulations via STAN run through version 4.0.2 of R \tom{}{You need to cite R and RStan}
For each model, we obtained 3 chains of 10,000 iterations, each with randomly chosen initial conditions. 
The first 1,500 iterations were discarded as burn-in to eliminate transience associated with initial conditions. 
We did not thin the chains, thus all samples were retained. 
To assess the convergence of our models we assessed between and within chain convergence, the resulting figures are included in supplemental documents. 
To assess the overall model fit we carried out posterior predictive checks to examine how well the fitted model can generate simulated data similar to the real data.
Large differences in the two indicate a poor model fit and can be assessed visually (figures included in supplemental documents). 
\tom{All estimated parameters are described in table 1.}{I think I suggested this years ago but there are so many parameters in your model that I don't think it is particularly useful to report all of these in a table. Including publicly available data and code is good enough.}
Data and code for all vital rate models is included in the supplemental information.

\subsection*{Integral Projection Model Construction}
  
Integral Projection Models describe population dynamics in discrete time, with functions that relate vital rates to continuous state variables. 
While IPMs are a natural choice for populations with continuous size structure, they can also be modified to accommodate a combination of continuous and discrete state variables, as we do here. 
We constructed a composite IPM which allows us to analyze the long term population growth rate of cholla with ant transition dynamics explicitly included.
%Many of the vital rate models are dependent not only on continuous size variables, but also on discrete variables representing the ant partner present.
%There is also a transition rate which determines what proportion of cholla are tended by each ant in a given year based on the previous ant partner and the size of the plants. 
This novel structure allows us to determine the individual effects of each ant species as well as the composite effects of several partners across the cholla population. 

Following previous studies of this population, we modeled the life cycle of cholla using continuously size-structured plants, $n_i(x)$, where $x$ is the size of the cholla and $i$ is the ant partner, and two discrete seed banks ($B_{1,t}$ and $B_{2,t}$) corresponding to 1 and 2-year old seeds.
The dynamics of these 1 and 2-year old seedbanks are given by the following equations: \tom{}{Rather than ``hard-code'' 4 in the summation, I would create a parameter for number of possible ant states, maybe $n_{A}$.}

  \begin{linenomath*}
		$$
		B_{1, t+1} = \kappa \delta \sum_{i}^{4} \int_L^U P(x) V_i(x) F(x) n_i(x) dx \\
		$$
		$$
		B_{2,t+1} =  (1 - \gamma_1)B_{1,t}\\
		$$
  \end{linenomath*}

The functions $P(x)$ and $F(x)$ give the probability of flowering, the number of flowerbuds produced based on the plant size $x$ and the year $t$. 
The proportion of flowerbuds which will produce seeds ($V_i(x)$) is dependent on the plant size $x$ and the ant species present on the plant $i$ in year $t$. 
The integral is multiplied by the number of seeds per fruit ($\kappa$) and the probability of seed dispersal/survival ($\delta$) to give the number of seeds that enter the one-year old seed bank. 
Parameters $U$ and $L$ are the upper and lower bounds, respectively, of the plant size distribution. 
Plants can recruit out of the one-year seed bank with the probability of $\gamma_1$ or transition to the two-year seed bank with a probability of $1 - \gamma_1$. 
Seeds in the two-year seed bank are assumed to either germinate with a probability of $\gamma_2$ or die. 
		
The \tom{size dynamics}{It is not just size dynamics, it is size and ant dynamics. That needs to be communicated more clearly.} of the cholla are given by:
		
	\begin{linenomath*}
		$$
		n(y,i)_{t+1} = (\gamma_1 B_{1,t} + \gamma_2 B_{2,t}) \eta(y) \omega \beta_i  + \\
		$$
		$$
		\sum_{j}^{4} \int_L^U S_j(x) G_j(y,x) \tau_{ij}(x) n_j(x) dx \\
		$$
	\end{linenomath*}
	
\tom{The first term gives the recruitment from one and two-year seed banks to a plant of size $y$, where $\eta(y) ~ N(***)$ gives the seedling size distribution and $\omega ~ ***$ gives the proportion of seedlings which surive from germination (late summer) to the census (May).
The second term reflects the changes in the population of the cholla which are not recruits, where $S_j(x)$ gives the probability of a plant of size $x$ in year $t$ surviving to year $t+1$ with partner $j$.
$G_j(y,x)$ gives the probability of growing from size $x$ in year $t$ to size $y$ in year $t+1$, respectively with partner $j$. 
Finally, $\tau_{ij}$ is the probability of a cholla which is size $x$ with ant partner $i$ in year $t$ being tended by ant partner $j$ in year $t+1$.}{This is a good start but I think you need a more thorough description, since this model will be unfamiliar to many readers.}

\subsection*{Deterministic IPM Analysis}
  \tom{}{The model you show above is the deterministic model. The stochastic model is a different model with additional parameters that should be represented symbolically. So my point is that I would no introduce the model first and then describe deterministic and stochastic analysis. Each of those thngs corresponds to a different model.}
Analyzing an IPM requires discretizing the composite IPM into a matrix to calculate the dominant eigenvalue. 
\tom{Each component of the IPM (each set of ant partner specific vital rates) }{I don't think readers will understand what you are talking about here} can be discretized into its own $ 200 \times 200$ matrix as shown in figure ***. 
The possibility of eviction, when individuals are predicted to grow outside of the possible size classes, was avoided by adding probabilities of growing smaller or larger than the existing boundaries, as done by many others***.
We used the composite IPM to quantify the effects of partner diversity on the intrinsic growth rate of cholla, $r$ $ln(\lambda)$. 
We calculated the $r$ for each combination of ant partners: complete vacancy; \textit{L. apiculatum} and vacancy; \textit{C. opuntiae} and vacancy; other and vacancy; \textit{L. apiculatum}, \textit{C. opuntiae}, and vacancy, \textit{L. apiculatum}, other, and vacancy; \textit{C. opuntiae}, other, and vacancy; and all ant partners and vacancy.
The \tom{relative distributions}{Unclear what this means} allowed us to determine if either complementarity or sampling effect were at play in the tree cholla-ant defense system.
\tom{Sampling Effect requires}{I don't think this is the right grammar (``requires'')} that the $r$ when all possible partners are present is \tom{equal to}{What if it is less than?} the $r$ of the cholla population when only the best partner is present.
Complementarity requires that the $r$ when all possible partners are present is greater than the $r$ of any other combination of partners or any single partner. 
\tom{}{These last two sentences are very important -- laying out predictions under alternative hypotheses -- so it is work making them laser sharp. }

\subsection*{Stochastic IPM Analysis}
  
The third mechanism considered in this study requires annual variation to be explicitly considered in the IPM so we constructed a stochastic version of the IPM described above. 
The year of record was a random effect in all vital rate models meaning we were able to include an intercept for specific years. 
In order to include stochasticity in the analysis,\tom{ we randomly sampled from the year-random effect intercepts 1,000 times}{I don't think you have communicated, in either words or notation, that the random intercepts are unique to ant state.}.
For each of these 1,000 iterations we calculated the $r$ for every possible combination of ant partners. 
With these values we calculated $\delta r$ between the scenario with all possible partners and the scenario with no partners.
This $\delta r$ measures the net benefits offered by partner diversity with annual variation.
We also calculated the $\delta r$ of the deterministic IPM to measure the net benefits offered by partner diversity without annual variation. 
These two relative differences allow us to determine if portfolio effect is at play.
\tom{Portfolio effect is demonstrated when the net benefit with annual variation is larger than the net benefit without annual variation.}{This is good and would make a good figure (posteriors of delta r with and without annual variation.)} 
			
    
  
\section*{Results}

\section*{Discussion}


\section*{Acknowledgments}

%%%%%%%%%%%%%%%%%%%%%
% Statement of Authorship
%%%%%%%%%%%%%%%%%%%%%
% This section should also be commented out while your MS is undergoing
% double-blind review. The specifics should of course be adapted to
% your paper, but the paragraph below gives some hints of possible
% contributions.

\section*{Data and Code Availability}

\section*{Appendix A: Additional Methods and Parameters}

% In most cases, authors should typeset supplementary material in a separate,
% author-supplied PDF. For author-supplied PDFs, please consult the
% AmNat_supp_template.tex document, available from
% https://www.journals.uchicago.edu/journals/an/instruct 
%
% By contrast, the Appendix instructions below apply to cases in which
% a brief appendix is to appear in print after the main body of the article.
% That notably includes descriptions of methods, tables defining parameters,
% and other material necessary for reproducing the MS's results.
%
% Please reset counters for the appendix (thus normally figure A1, 
% figure A2, table A1, etc.).
%
% Most AmNat articles have no more than one print appendix. If your article
% has more than one, counters for each appendix should match the letter of
% that appendix. For example, tables in Appendix B should be numbered table B1, % table C2, etc. This applies to tables, equations, and figures.
%
% It's better not to use the \appendix command, because we have some
% formatting peculiarities that \appendix conflicts with.

\renewcommand{\theequation}{A\arabic{equation}}
% redefine the command that creates the equation number.
\renewcommand{\thetable}{A\arabic{table}}
\setcounter{equation}{0}  % reset counter 
\setcounter{figure}{0}
\setcounter{table}{0}

%%%%%%%%%%%%%%%%%%%%%
% Bibliography
%%%%%%%%%%%%%%%%%%%%%
% You can either type your references following the examples below, or
% compile your BiBTeX database and paste the contents of your .bbl file
% here. The amnatnat.bst style file should work for this---but please
% let us know if you run into any hitches with it!
%
% If you upload a .bib file with your submission, please upload the .bbl
% file as well; this will be required for typesetting.
%
% The list below includes sample journal articles, book chapters, and
% Dryad references.
\bibliographystyle{apalike}
\bibliography{References.bib}


\newpage{}

\section*{Tables}
\renewcommand{\thetable}{\arabic{table}}
\setcounter{table}{0}

\renewcommand{\thetable}{\arabic{table}}
\setcounter{table}{0}

% I am creating a table here to include all parameter estimations and descriptions
  \begin{table}[]
  \begin{tabular}{l|l|l}
    \textbf{Parameter} & \textbf{Median ($95\%$ CI)} & \textbf{Prior Distribution} \\
    \hline
    %% Growth Parameters
    growth xi intercept vacant $\Beta_{01}^g$ & -5.210899 (-5.686865, -5.491787) & \\
    growth xi intercept other $\Beta_{02}^g$ & -5.8288 (-5.956217, 1.766021) & \\
    growth xi intercept \textit{C. opuntiae} $\Beta_{03}^g$ & -4.529523 (-6.0770390, 0.1222112) & \\
    growth xi intercept \textit{L. apiculatum} $\Beta_{04}^g$ & -5.106802 (-5.4499944, 0.5453901) & \\
    growth xi size dependent vacant $\Beta_{11}^g$ & & \\
    growth xi size dependent other $\Beta_{12}^g$ & & \\
    growth xi size dependent \textit{C. opuntiae} $\Beta_{13}^g$ & & \\
    growth xi size dependent \textit{L. apiculatum} $\Beta_{14}^g$ & & \\
    growth omega intercept $\omega_0^g$ & & \\
    growth omega size dependent $\omega_1^g$ & & \\
    growth alpha intercept $\alpha_0^g$ & & \\
    growth alpha size dependent $\alpha_1^g$ & & \\
    \hline
    %% Germination Parameters
    1-year germination intercept $\alpha^{\gamma_1}$ & & \\
    2-year germination intercept $\alpha^{\gamma_2}$ & & \\
    \hline
    %% Survival Parameters
    survival intercept vacant $\Beta_{01}^s$ & & \\
    survival intercept other $\Beta_{02}^s$ & & \\
    survival intercept \textit{C.opuntiae} $\Beta_{03}^s$ & & \\
    survival intercept \textit{L. apiculatum} $\Beta_{04}^s$ & & \\
    survival size dependent vacant $\Beta_{11}^s$ & & \\
    survival size dependent other $\Beta_{12}^s$ & & \\
    survival size dependent \textit{C. opuntiae} $\Beta_{13}^s$ & & \\
    survival size dependent \textit{L. apiculatum} $\Beta_{14}^s$ & & \\
    \hline
    %% Probability of Flowering Parameters
    flowering intercept $\Beta_0^f$ & & \\
    flowering size dependent $\Beta_1^f$ & & \\
    \hline
    %% Floral Viability Parameters
    viability intercept vacant $\Beta_01^v$ & & \\
    viability intercept other $\Beta02^v$ & & \\
    viability intercept \textit{C. opuntiae} $\Beta_03^v$ & & \\
    viability intercept \textit{L. apiculatum} $\Beta_04^v$ & & 
  \end{tabular}
  \end{table}
\bigskip{}
\\
\end{table}

\newpage{}

\section*{Figure legends}


%%%%%%%%%%%%%%%%%%%%%
% Videos
%%%%%%%%%%%%%%%%%%%%%
% If you have videos, journal style for them is generally similar to that for
% figures. 

%%%%% Include the text below if you have videos



%%%%% Include the above if you have videos


\end{document}
