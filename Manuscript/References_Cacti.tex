@article{Jones2015,
abstract = {Cheating is a focal concept in the study of mutualism, with the majority of researchers considering cheating to be both prevalent and highly damaging. However, current definitions of cheating do not reliably capture the evolutionary threat that has been a central motivation for the study of cheating. We describe the development of the cheating concept and distill a relative-fitness-based definition of cheating that encapsulates the evolutionary threat posed by cheating, i.e. that cheaters will spread and erode the benefits of mutualism. We then describe experiments required to conclude that cheating is occurring and to quantify fitness conflict more generally. Next, we discuss how our definition and methods can generate comparability and integration of theory and experiments, which are currently divided by their respective prioritisations of fitness consequences and traits. To evaluate the current empirical evidence for cheating, we review the literature on several of the best-studied mutualisms. We find that although there are numerous observations of low-quality partners, there is currently very little support from fitness data that any of these meet our criteria to be considered cheaters. Finally, we highlight future directions for research on conflict in mutualisms, including novel research avenues opened by a relative-fitness-based definition of cheating.},
author = {Jones, Emily I. and Afkhami, Michelle E. and Ak{\c{c}}ay, Erol and Bronstein, Judith L. and Bshary, Redouan and Frederickson, Megan E. and Heath, Katy D. and Hoeksema, Jason D. and Ness, Joshua H. and Pankey, M. Sabrina and Porter, Stephanie S. and Sachs, Joel L. and Scharnagl, Klara and Friesen, Maren L.},
doi = {10.1111/ele.12507},
file = {:Users/alicampbell/Box Sync/Grad/Cactus/Papers/Jones_2015.pdf:pdf},
issn = {14610248},
journal = {Ecology Letters},
keywords = {Ant-plant,Cleaner fish-client,Cooperation,Fig-fig wasp,Fitness conflict,Legume-rhizobia,Nectar larceny,Partner quality,Plant-mycorrhizae,Yucca-yucca moth},
number = {11},
pages = {1270--1284},
title = {{Cheaters must prosper: Reconciling theoretical and empirical perspectives on cheating in mutualism}},
volume = {18},
year = {2015}
}
@article{Bronstein2001,
abstract = {Mutualisms arc of central importance in biological systems. Despite growing attention in recent years, however, few conceptual themes have yet to be identified that span mutualisms differing in natural history. Here I examine the idea that the ecology and evolution of mutualisms are shaped by diverse costs, not only by the benefits they confer. This concept helps link mutualism to antagonisms such as herbivory, pr{\'{e}}dation, and parasitism, interactions defined largely by the existence of costs. I first briefly review the range of costs associated with mutualisms, then describe how one cost, the consumption of seeds by pollinator offspring, was quantified for one fig/pollinator mutualism. I compare this cost to published values for other fig/pollinator mutualisms and for other kinds of pollinating seed parasite mutualisms, notably the yucca/yucca moth interaction. I then discuss four issues that fundamentally complicate comparative studies of the cost of mutualism:. problems of knowing how to measure the magnitude of any one cost accurately; problems associated with using average estimates in the absence of data on sources of variation; complications arising from the complex correlates of costs, such as functional linkages between costs and benefits; and problems that arise from considering the cost of mutualism as a unilateral issue in what is fundamentally a reciprocal interaction. The rich diversity of as-yet unaddressed questions surrounding the costs of mutualism may best be investigated via detailed studies of individual interactions.},
author = {Bronstein, Judith L.},
doi = {10.1093/icb/41.4.825},
file = {:Users/alicampbell/Box Sync/Grad/Cactus/Papers/Bronstein2001.pdf:pdf},
issn = {00031569},
journal = {American Zoologist},
number = {4},
pages = {825--839},
title = {{The costs of mutualism}},
volume = {41},
year = {2001}
}
@article{Bronstein2001a,
author = {Bronstein, Judith L.},
journal = {Ecology Letters},
number = {3},
pages = {277--287},
title = {{The Exploitation of Mutualisms}},
volume = {4},
year = {2001}
}
@article{Bronstein2006,
author = {Bronstein, Judith L and Bronstein, Judith L and Alarc{\'{o}}n, Ruben and Geber, Monica},
file = {:Users/alicampbell/Box Sync/Grad/Cactus/Papers/Bronstein 2006.pdf:pdf},
pages = {412--428},
title = {{The evolution of plant – insect mutualisms}},
year = {2006}
}
@article{Bronstein1998,
author = {Bronstein, Judith L.},
journal = {bioTropica},
number = {2},
pages = {150--161},
title = {{The Contribution of Ant-Plant Protection Studies to Our Understanding of Mutualism}},
volume = {30},
year = {1998}
}
@article{Bascompte2019,
abstract = {Mutualism is a type of interaction in which both partners benefit from each other. For example, a butterfly receives nectar, a rich source of food, from the flower of a plant and in turn moves pollen from that plant to another far away (Figure 1). In order to reflect about the widespread nature of mutualism, John N. Thompson proposed the following thought experiment: try to imagine a plant species that is viable in its natural habitat without using on top of its own nuclear genome the genomes of a mitochondrion, a chloroplast, of several mycorrhizal fungi, of several insects to pollinate it and of several species of bird to disperse its seeds. Mutualism is everywhere and it is assumed that mutualistic interactions have played a major role in the diversification of life on Earth. An often-cited example is the rich adaptive radiation of the flowering plants (angiosperms), with about 300,000 described species. Flowering plants originated around 160 million years ago and diversified fast during the Early Cretaceous, so that by around 120 million years ago they had become widespread. It is generally assumed that such a fast diversification is largely the result of the mutualistic interaction with pollinators. Thus, mutualism has been most likely shaping the diversity of species on Earth from an early stage. But the relationships between mutualism and diversity are not yet clear, mainly because mutualism has traditionally been studied within pairs or small groups of species. Also, mutualism has historically been studied in isolation from competition, so it is unclear how these two forces balance each other in ecological communities. Jordi Bascompte introduces mutualistic networks and how they affect biodiversity patterns.},
author = {Bascompte, Jordi},
doi = {10.1016/j.cub.2019.03.062},
file = {:Users/alicampbell/Box Sync/Grad/Cactus/Papers/Bascompote_.pdf:pdf},
issn = {09609822},
journal = {Current Biology},
number = {11},
pages = {R467--R470},
pmid = {31163160},
publisher = {Elsevier},
title = {{Mutualism and biodiversity}},
url = {http://dx.doi.org/10.1016/j.cub.2019.03.062},
volume = {29},
year = {2019}
}
@article{Stachowicz2001,
abstract = {POSITIVE INTERACTIONS PLAY A CRITICAL, BUT UNDERAPPRECIATED, ROLE IN ECOLOGICAL COMMUNITIES BY REDUCING PHYSICAL OR BIOTIC STRESSES IN EXISTING HABITATS AND BY CREATING NEW HABITATS ON WHICH MANY SPECIES DEPEND},
author = {Stachowicz, John J.},
file = {:Users/alicampbell/Box Sync/Grad/Cactus/Papers/Stachowicz_2001.pdf:pdf},
issn = {0006-3568},
journal = {BioScience},
number = {3},
pages = {235--246},
title = {{the Structure of Ecological Communities}},
volume = {51},
year = {2001}
}
@article{Chamberlain2014,
abstract = {The net effects of interspecific species interactions on individuals and populations vary in both sign (-, 0, +) and magnitude (strong to weak). Interaction outcomes are context-dependent when the sign and/or magnitude change as a function of the biotic or abiotic context. While context dependency appears to be common, its distribution in nature is poorly described. Here, we used meta-analysis to quantify variation in species interaction outcomes (competition, mutualism, or predation) for 247 published articles. Contrary to our expectations, variation in the magnitude of effect sizes did not differ among species interactions, and while mutualism was most likely to change sign across contexts (and predation least likely), mutualism did not strongly differ from competition. Both the magnitude and sign of species interactions varied the most along spatial and abiotic gradients, and least as a function of the presence/absence of a third species. However, the degree of context dependency across these context types was not consistent among mutualism, competition and predation studies. Surprisingly, study location and ecosystem type varied in the degree of context dependency, with laboratory studies showing the highest variation in outcomes. We urge that studying context dependency per se, rather than focusing only on mean outcomes, can provide a general method for describing patterns of variation in nature. {\textcopyright} 2014.},
author = {Chamberlain, Scott A. and Bronstein, Judith L. and Rudgers, Jennifer A.},
doi = {10.1111/ele.12279},
file = {:Users/alicampbell/Box Sync/Grad/Endophytes/Papers/Chamberlain_2014.pdf:pdf},
issn = {14610248},
journal = {Ecology Letters},
keywords = {Coefficient of variation,Community context,Conditionality,Distributed outcomes,Interaction strength,Meta-analysis},
number = {7},
pages = {881--890},
pmid = {24735225},
title = {{How context dependent are species interactions?}},
volume = {17},
year = {2014}
}
@incollection{Winfree2020,
author = {Winfree, Rachel},
booktitle = {Unsolved Problems in Ecology},
pages = {337--354},
publisher = {Princeton University Press},
title = {{How Does Biodiversity Relate to Ecosystem Functioning in Natural Ecosystems?}},
year = {2020}
}
@article{West2007,
author = {West, Stuart A. and Griffin, Ashleigh S. and Gardner, Andy},
journal = {Current Biology},
number = {16},
pages = {R661--R672},
title = {{Evolutionary Explanations for Cooperation}},
volume = {17},
year = {2007}
}
@article{Stanton2013,
author = {Stanton, Maureen L.},
journal = {The American Naturalist},
pages = {510--523},
title = {{Interacting Guilds: Moving beyond the Pairwise Perspective on Mutualisms}},
volume = {162},
year = {2013}
}
@article{Stachowicz2005,
author = {Stachowicz, John J. and Whitlatch, Robert B.},
journal = {Ecology},
number = {9},
pages = {2418--2427},
title = {{Multiple mutualists provide complementary benefits to their seaweed host}},
volume = {86},
year = {2005}
}
@article{Song2020,
author = {Song, Chuliang and Ahn, Sarah Von and Rohr, Rudolf P. and Saavedra, Serguei},
journal = {Trends in Ecology and Evolution},
number = {5},
pages = {384--396},
title = {{Towards Probabilistic Understanding About the Context-Dependency of Species Interactions}},
volume = {35},
year = {2020}
}
@article{Rodriguez-Rodriguez2017,
author = {Rodriguez-Rodriguez, Maria C. and Pedro, Jordano and Valido, Alfredo},
journal = {Ecology},
number = {5},
pages = {1266--1276},
title = {{Functional consequences of plant-animal interactions along the mutualism-antagonism gradient}},
volume = {98},
year = {2017}
}
@article{Palmer2010,
author = {Palmer, Todd M. and Doak, Daniel F. and Stanton, Maureen L. and Bronstein, Judith L. and Kiers, Toby E. and Young, Truman P. and Goheen, Jacob R. and Pringle, Robert M.},
journal = {PNAS},
number = {40},
pages = {17234--17239},
title = {{Synergy of multiple partners, including freeloaders, increases host fitness in a multispecies mutualism}},
volume = {107},
year = {2010}
}
@article{Ohm2014,
annote = {These plants have tradeoffs over time between the Crem and Liom -equal at young -Liom only at reproduction The ants have different defense abilities and pollinator tradeoffs -liom does not have as big of negative effects on pollinators and does a better job defending from enemies -crem has big pollination effects (negative) as well as does a worse job defending -part of the reason why this is possible is that the ants are active at different times of day than the pollinators The Liom are the better ant defender because they keep away more enemies as well has have less negative impacts on the reproduction of the plants (which is the most important mechanism for fitness). The Crem are good enough for young plants because pollination is not important and it has been shown in different papers that having any defender is better than having no defender It appears that the presence of aggressive defenders has little impact on pollination.},
author = {Ohm, Johanna R. and Miller, Tom E.X.},
journal = {Ecology},
mendeley-groups = {Cactus Info},
number = {10},
pages = {2924--2935},
title = {{Balancing Anti-Herbivore Benefits and Anti-Pollinator Costs of Defensive Mutualists}},
volume = {95},
year = {2014}
}
@article{Miller2007,
author = {Miller, Tom E.X.},
journal = {Oikos},
number = {3},
pages = {500--512},
title = {{Does Having Multiple Partners Weaken the Benefits of Faculative Mutualism? A Test with Cacti and Cactus-Tending Ants}},
volume = {116},
year = {2007}
}
@article{Miller2014,
abstract = {Many plants engage ants in defensive mutualisms by offering extrafloral nectar (EFN). Identifying sources of variation in EFN quantity (amount) and quality (composition) is important because they can affect ant visitation and identity and hence effectiveness of plant defence. I investigated plant size and reproductive state (vegetative or flowering) as sources of variation in EFN quantity and quality. I focused on Opuntia imbricata and two ant partners, Crematogaster opuntiae and Liometopum apiculatum. I tested the influence of plant size and nectary type (vegetative vs. reproductive structure) on the probability and rate of EFN secretion, concentrations of total carbohydrates (CH) and amino acids (AAs), and relative abundances of constituent CH and AAs. I also examined how traits of individual nectaries scaled up to influence total plant-level rewards. Parallel observations documented associations between plant demographic state and ant visitation and species identity. EFN quantity and quality were generally greater for larger, reproductive plants. At the scale of individual nectaries, probability of EFN secretion was positively size-dependent and greater for nectaries on reproductive vs. vegetative structures. Rate of EFN secretion, carbohydrate and amino acid concentrations, and the relative abundance of disaccharide vs. monosaccharide sugars were greater for reproductive nectaries but were unaffected by plant size. Nectary-level traits scaled up to influence rewards at the whole-plant level in ways that corresponded to ant visitation: the probability of ant occupancy increased with plant size and reproduction, as did the likelihood of being tended by the superior guard, L. apiculatum. Variability in EFN traits may contribute to changes in ant occupancy and identity across plant sizes and reproductive states. Synthesis. This study provides a thorough examination of how plant investment in biotic defence varies over the life cycle. Explicit consideration of plant demography may enhance understanding of ant-plant mutualisms. Populations of long-lived plants are demographically heterogeneous, spanning sizes and reproductive states. The rewards offered to animal mutualists can track demographic heterogeneity with consequences for plant defence and the dynamics of multispecies mutualisms. This study provides a thorough examination of how plant investment in biotic defence varies over the life cycle. Explicit consideration of plant demography may enhance understanding of ant-plant mutualisms. Populations of long-lived plants are demographically heterogeneous, spanning sizes and reproductive states. The rewards offered to animal mutualists can track demographic heterogeneity with consequences for plant defence and the dynamics of multispecies mutualisms. {\textcopyright} 2013 British Ecological Society.},
author = {Miller, Tom E.X.},
doi = {10.1111/1365-2745.12210},
file = {:Users/alicampbell/Box Sync/Grad/Cactus/Papers/Miller 2014.pdf:pdf},
issn = {00220477},
journal = {Journal of Ecology},
keywords = {Ant-plant mutualism,Biotic defence,Demography,Extrafloral nectar,Invertase,Ontogeny,Plant development and life-history traits,Stage structure},
mendeley-groups = {Cactus Info},
number = {2},
pages = {496--507},
title = {{Plant size and reproductive state affect the quantity and quality of rewards to animal mutualists}},
volume = {102},
year = {2014}
}
@article{Benson1982,
author = {Benson, L.},
journal = {Stanford University Press, Stanford, CA.},
mendeley-groups = {Cactus Info},
title = {{Cacti of the United States and Canada}},
year = {1982}
}
@article{Ness2006,
author = {Ness, Joshua H. and Morris, W.F. and Bronstein, Judith L.},
journal = {Ecology},
mendeley-groups = {Cactus Info},
number = {4},
pages = {912--921},
title = {{INTEGRATING QUALITY AND QUANTITY OF MUTUALISTIC SERVICE TO CONTRAST ANT SPECIES PROTECTING FEROCACTUS WISLIZENI}},
volume = {87},
year = {2006}
}
@article{Oliveira1999,
abstract = {1. This study examines the anti-herbivore effect of ants visiting the extrafloral nectaries (EFNs) of Opuntia stricta (Cactaceae) and its possible influence on the plant's reproductive output in Mexican coastal sand dunes. Opuntia's EFNs are located in the areoles of the developing tissue of emerging cladodes and flower buds. 2. Ants visited the EFNs of O. stricta on a round-the-clock basis. The associated ant assemblage was formed by nine species distributed in four subfamilies, and the species composition of the principal ant visitors changed markedly from clay to night period. 3. Cladodes of control (ants present) and treatment (ants excluded) plants of Opuntia were equally infested by sucking bugs and mining dipterans. Damage to buds by a pyralid moth, however, was significantly higher on treatment than on control plants. Ant visitation to Opuntia's EFNs translated into a 50% increase in the plant's reproductive output, as expressed by the number of fruits produced by experimental control and treatment branches. Moreover, fruit production by ant-visited branches was positively and significantly associated with the mean monthly rate of ant visitation to EFNs. 4. This is the first demonstration of ant protection leading to increased fruit set in the Cactaceae under natural conditions. Although the consequences of damage by sucking and mining insects remain unclear for Opuntia, the results show how the association of EFNs with vulnerable reproductive plant organs can result in a direct ant-derived benefit to plant fitness.},
author = {Oliveira, P. S. and Rico-Gray, V. and D{\'{i}}az-Castelazo, C. and Castillo-Guevara, C.},
doi = {10.1046/j.1365-2435.1999.00360.x},
file = {:Users/alicampbell/Box Sync/Grad/Endophytes/Papers/Oliveira and Rico-Gray etal. 1999.pdf:pdf},
issn = {02698463},
journal = {Functional Ecology},
keywords = {Ant foraging schedule,Ant protection,Ant-plant mutualism,Cactus,Herbivory,Reproductive output},
mendeley-groups = {Cactus Info},
number = {5},
pages = {623--631},
title = {{Interaction between ants, extrafloral nectaries and insect herbivores in Neotropical coastal sand dunes: Herbivore deterrence by visiting ants increases fruit set in Opuntia stricta (Cactaceae)}},
volume = {13},
year = {1999}
}
@book{Lach2010,
author = {Lach, Lori and Parr, Catherine L. and Abbott, Kirsti L.},
mendeley-groups = {Cactus Info},
pages = {1--23,75--137,157--230},
title = {{Ant Ecology}},
year = {2010}
}
@article{Mann1969,
author = {Mann, J},
file = {:Users/alicampbell/Box Sync/Grad/Cactus/Papers/Mann 1969.pdf:pdf},
journal = {Smithsonian Inst.},
mendeley-groups = {Cactus Info},
title = {{Cactus-feeding insects and mites.}},
year = {1969}
}
@article{Miller2013,
annote = {This paper shows that the benefits to the ants in our system come primarily in nectar production. The nectar production, however can change in quantity and quality. Reproductive state influences both quantity and quality of the EFN. The probability of EFN secretion in increased by size. This indicates that potentially there is a feedback with ant species and these factors for the EFN.},
author = {Miller, Tom E.X.},
file = {:Users/alicampbell/Box Sync/Grad/Cactus/Papers/Miller_2014_JEcol (2).pdf:pdf},
journal = {Journal of Ecology},
mendeley-groups = {Cactus Info},
pages = {1--11},
title = {{Plant size and reproductive state affect the quantity and quality of rewards to animal mutualists}},
year = {2013}
}
@article{Miller2006,
abstract = {Life-history theory suggests that iteroparous plants should be flexible in their allocation of resources toward growth and reproduction. Such plasticity could have consequences for herbivores that prefer or specialize on vegetative versus reproductive structures. To test this prediction, we studied the response of the cactus bug (Narnia pallidicornis) to meristem allocation by tree cholla cactus (Opuntia imbricata). We evaluated the explanatory power of demographic models that incorporated variation in cactus relative reproductive effort (RRE; the proportion of meristems allocated toward reproduction). Field data provided strong support for a single model that denned herbivore fecundity as a time-varying, increasing function of host RRE. High-RRE plants were predicted to support larger insect populations, and this effect was strongest late in the season. Independent field data provided strong support for these qualitative predictions and suggested that plant allocation effects extend across temporal and spatial scales. Specifically, late-season insect abundance was positively associated with interannual changes in cactus RRE over 3 years. Spatial variation in insect abundance was correlated with variation in RRE among five cactus populations across New Mexico. We conclude that plant allocation can be a critical component of resource quality for insect herbivores and, thus, an important mechanism underlying variation in herbivore abundance across time and space. {\textcopyright} 2006 by The University of Chicago.},
author = {Miller, Tom E.X. and Tyre, Andrew J. and Louda, Svata M.},
doi = {10.1086/509610},
file = {:Users/alicampbell/Box Sync/Grad/Cactus/Papers/Miller etal 2006.pdf:pdf},
issn = {00030147},
journal = {American Naturalist},
keywords = {Demographic model,Herbivory,Insect-plant interactions,Opuntia,Population dynamics,Resource allocation},
mendeley-groups = {Cactus Info},
number = {5},
pages = {608--616},
pmid = {17080360},
title = {{Plant reproductive allocation predicts herbivore dynamics across spatial and temporal scales}},
volume = {168},
year = {2006}
}
@article{Miller2009,
abstract = {Understanding the role of consumers in plant population dynamics is important, both conceptually and practically. Yet, while the negative effects of herbivory on plant performance have been well documented, we know much less about how individuallevel damage translates to impacts on population growth or whether spatial variation in herbivory affects patterns of plant distribution. We studied the role of insect herbivory in the dynamics and distribution of the tree cholla cactus (Opuntia imbricata), a long-lived perennial plant, across an elevational gradient in central New Mexico, USA, from low-elevation grassland (1670 m) to a grassland-mountain transition zone (1720 m) to the rocky slopes of the Los Pinos Mountains (1790 m). Tree cholla density increased significantly with elevation, while abundance of and damage by a suite of native, cactus-feeding insects decreased. We combined field experiments and demographic models to test the hypothesis that systematic spatial variation in chronic insect herbivory limits the tree cholla distribution to a subset of suitable habitat across the gradient. Our results support this hypothesis. We found that key demographic functions (survival, growth, fecundity) and the responses of these functions to experimental reductions in insect herbivory varied across the gradient. The effects of insect exclusion on plant growth and seed production were strongest in the lowelevation grassland and decreased in magnitude with increasing elevation. We used the experimental data to parameterize integral projection models (IPM), which predict the asymptotic rate of population increase ($\lambda$). The modeling results showed that insect herbivory depressed $\lambda$ and that the magnitude of this effect was context-dependent. The effect of insect herbivory on population growth was strongest at low elevation ($\delta$$\lambda$klow = 0.095), intermediate at mid elevation ($\delta$$\lambda$mid = 0.046), and weakest at high elevation ($\delta$$\lambda$high = -0.0089). The total effect of insects on $\lambda$ was due to a combination of reductions in growth and in fecundity and their combination; the relative contribution of each of these effects varied spatially. Our results, generated by experimental demography across a heterogeneous landscape, provide new insights into the role of native consumers in the population dynamics and distribution of abundance of long-lived native plants.},
author = {Miller, Tom E.X. and Louda, Svata M. and Rose, Karen A. and Eckberg, James O.},
doi = {10.1890/07-1550.1},
file = {:Users/alicampbell/Box Sync/Grad/Cactus/Papers/Miller etal 2009.pdf:pdf},
issn = {00129615},
journal = {Ecological Monographs},
keywords = {Cahela ponderosella,Chihuahuan Desert,Elevation gradient,Herbivory,Integral projection model,New Mexico,Opuntia imbricata,Plant-Insect interactions,Population dynamics,USA,cactus-feeding insects},
mendeley-groups = {Cactus Info},
number = {1},
pages = {155--172},
title = {{Impacts of insect herbivory on cactus population dynamics: Experimental demography across an environmental gradient}},
volume = {79},
year = {2009}
}
@article{Isbell2011,
author = {Isbell, Forest},
journal = {Nature},
pages = {199--202},
title = {{High plant diversity is needed to maintain ecosystem services}},
volume = {477},
year = {2011}
}
@article{Hooper2005,
author = {Hooper, D. U.},
journal = {Ecological Monographs},
number = {1},
pages = {3--35},
title = {{Effects of Biodiversity on Ecosystem Functioning: A Consensus of Current Knowledge}},
volume = {75},
year = {2005}
}
@article{Frederickson2013,
author = {Frederickson, Megan E.},
journal = {Quarterly Review of Biology},
number = {4},
pages = {269--295},
title = {{Rethinking Mutualism Stability: Cheaters and the Evolution of Sanctions}},
volume = {88},
year = {2013}
}
@article{Cardinale2012,
author = {Cardinale, Bradley J.},
journal = {American Journal of Botany},
number = {3},
pages = {572--592},
title = {{The Functional Role of Producer Diversity in Ecosystems}},
volume = {98},
year = {2012}
}
@article{Bronstein1994,
author = {Bronstein, Judith L.},
journal = {TREE},
number = {6},
pages = {214--217},
title = {{Conditional Outcomes in Mutualistic Interactions}},
volume = {9},
year = {1994}
}
@article{Boucher1982,
author = {Boucher, Douglas H.},
journal = {Annual Review of Ecological Systems},
pages = {317--347},
title = {{The Ecology of Mutualism}},
volume = {13},
year = {1982}
}
@article{Boege2005,
author = {Boege, Karina and Marquis, Robert J.},
journal = {Trends in Ecology and Evolution},
pages = {441--448},
title = {{Facing Herbivory as You Grow up: The Ontogeny of Resistance in Plants}},
volume = {20},
year = {2005}
}
@article{Batstone2018,
author = {Batstone, Rebecca T.},
journal = {Ecology},
number = {5},
pages = {1039--1050},
title = {{Using niche breadth theory to explain generalization in mutualisms}},
volume = {99},
year = {2018}
}
@article{Barton2010,
author = {Barton, Kasey E. and Koricheva, Julia},
journal = {The American Naturalist},
number = {4},
pages = {481--493},
title = {{The Ontogeny of Plant Defense and Herbivory: Characterizing General Patterns Using Meta-Analysis}},
volume = {175},
year = {2010}
}
@article{Amarasekare2003,
author = {Amarasekare},
journal = {Ecology Letters},
pages = {1109--1122},
title = {{Competitive coexistence in spatially structured environments: a synthesis}},
volume = {6},
year = {2003}
}
@article{Afkhami2014,
author = {Afkhami},
journal = {Ecology},
number = {4},
pages = {833--844},
title = {{Multiple mutualist effects: conflict and synergy in multispecies mutualisms}},
volume = {95},
year = {2014}
}
